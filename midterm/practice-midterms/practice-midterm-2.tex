\documentclass[10pt]{article}
% Math Packages
\usepackage{amsmath, mathtools}
\usepackage{amssymb}
\usepackage{amsthm}
\usepackage{amsfonts}
\usepackage{bbm}
\usepackage{breqn}
\usepackage[margin=1in]{geometry}
\usepackage{graphicx}
\usepackage{tikz}
\usetikzlibrary{arrows.meta}
\usetikzlibrary{calc}
\usepackage{forest}
\usepackage{tikz-qtree}
\graphicspath{ {./images/} }
\usepackage{hyperref}
\usepackage[capitalize]{cleveref}
\usepackage[shortlabels]{enumitem}
\usetikzlibrary{arrows,matrix,positioning}
\usepackage{multicol}

% for the pipe symbol
\usepackage[T1]{fontenc}

% Citing theorems by name. (source: https://tex.stackexchange.com/questions/109843/cleveref-and-named-theorems)
\makeatletter
\newcommand{\ncref}[1]{\cref{#1}\mynameref{#1}{\csname r@#1\endcsname}}

\def\mynameref#1#2{%
  \begingroup
  \edef\@mytxt{#2}%
  \edef\@mytst{\expandafter\@thirdoffive\@mytxt}%
  \ifx\@mytst\empty\else
  \space(\nameref{#1})\fi
  \endgroup
}
\makeatother

% Colorful Notes
\usepackage{color} \definecolor{Red}{rgb}{1,0,0} \definecolor{Blue}{rgb}{0,0,1}
\definecolor{Purple}{rgb}{.5,0,.5} \def\red{\color{Red}} \def\blue{\color{Blue}}
\def\gray{\color{gray}} \def\purple{\color{Purple}}
\newcommand{\rnote}[1]{{\red [#1]}} % \rnote{foo} gives '[foo]' in red
\newcommand{\pnote}[1]{{\purple [#1]}} % \pnote{foo} gives '[foo]' in purple
\newcommand{\bnote}[1]{{\blue #1}} % \bnote{foo} gives 'foo' in blue
\newcommand{\gnote}[1]{{\gray #1}} % \gnote{foo} gives 'foo' in gray
\newcommand{\Max}[1]{{\purple [#1]}} % \bnote{foo} then 'foo' is blue


% Claim numbering (the counter restarts after each proof environment)
\newcounter{claimcount}
\setcounter{claimcount}{0}
\newenvironment{claim}{\refstepcounter{claimcount}\par\addvspace{\medskipamount}\noindent\textbf{Claim \arabic{claimcount}:}}{}
\usepackage{etoolbox}
\AtBeginEnvironment{proof}{\setcounter{claimcount}{0}}
\newenvironment{claimproof}{\par\addvspace{\medskipamount}\noindent\textit{Proof of Claim  \arabic{claimcount}.}}{\hfill\ensuremath{\qedsymbol} \tiny{Claim}

  \medskip}
% Add claim support to cleverref
\crefname{claimcount}{Claim}{Claims}


% Math Environments
\newtheorem{theorem}{Theorem}
\newtheorem{assumption}[theorem]{Assumption}
\newtheorem{lemma}[theorem]{Lemma}
\newtheorem{proposition}[theorem]{Proposition}
\newtheorem{corollary}[theorem]{Corollary}
\newtheorem{question}[theorem]{Question}
\theoremstyle{definition}
\newtheorem{definition}[theorem]{Definition}
\newtheorem{remark}[theorem]{Remark}
\newtheorem{example}[theorem]{Example}
\newtheorem{notation}[theorem]{Notation}
\newtheorem{problem}[theorem]{Problem}

% Matrices and Column Vectors. 
\usepackage{stackengine}
\setstackgap{L}{1.0\normalbaselineskip}
\usepackage{tabstackengine}
\setstackEOL{;}% row separator
\setstackTAB{,}% column separator
\setstacktabbedgap{1ex}% inter-column gap
\setstackgap{L}{1.5\normalbaselineskip}% inter-row baselineskip
\let\nmatrix\bracketMatrixstack  %Usage: \nmatrix{1,2,3\4,5,6}
\newcommand\cv[1]{\setstackEOL{,}\bracketMatrixstack{#1}} %usage: \cv{1,2,3}

% Custom Math Coqmmands
\newcommand{\vt}{\vskip 5mm} % vertical space
\newcommand{\fl}{\noindent\textbf} % first line
\newcommand{\Fl}{\vt\noindent\textbf} % first line with space above
\newcommand{\norm}[1]{\left\lVert#1\right\rVert} % norm
\newcommand{\pnorm}[1]{\left\lVert#1\right\rVert_p} % p-norm
\newcommand{\qnorm}[1]{\left\lVert#1\right\rVert_q} % q-norm
\newcommand{\1}[1]{\textbf{1}_{\left[#1\right]}} % indicator function 
\def\limn{\lim_{n\to\infty}} % shortcut for lim as n-> infinity
\def\sumn{\sum_{n=1}^{\infty}} % shortcut for sum from n=1 to infinity
\def\sumkn{\sum_{k=1}^{n}} % shortcut for sum from k=1 to n
\def\sumin{\sum_{i=1}^{n}} % shortcut for sum from i=1 to n
\def\SAs{\sigma\text{-algebras}} % shortcut for $\sigma$-algebras
\def\SA{\sigma\text{-algebra}} % shortcut for $\sigma$-algebra
\def\Ft{\mathcal{F}_t} % time-indexed sigma-algebra (t)
\def\Fs{\mathcal{F}_s} % time-indexed sigma-algebra (s)
\def\F{\mathcal{F}} % sigma-algebra
\def\G{\mathcal{G}} % sigma-algebra
\def\R{\mathbb{R}} % Real numbers
\def\N{\mathbb{N}} % Natural numbers
\def\Z{\mathbb{Z}} % Integers
\def\E{\mathbb{E}} % Expectation
\def\P{\mathbb{P}} % Probability
\def\Q{\mathbb{Q}} % Q probability
\def\dist{\text{dist}} %Text 'dist' for things like 'dist(x,y)'
\newcommand{\indep}{\perp \!\!\! \perp}  %independence symbol
\def\Var{\mathrm{Var}} % Variance
\def\tr{\mathrm{tr}} % trace

% Brackets and Parentheses
\def\[{\left [}
    \def\]{\right ]}
% \def\({\left (}
%   \def\){\right )}



\usepackage{color}
\definecolor{Red}{rgb}{1,0,0}
\definecolor{Blue}{rgb}{0,0,1}
\definecolor{Purple}{rgb}{.5,0,.5}
\def\red{\color{Red}}
\def\blue{\color{Blue}}
\def\gray{\color{gray}}
\def\purple{\color{Purple}}
\definecolor{RoyalBlue}{cmyk}{1, 0.50, 0, 0}
\newcommand{\dempfcolor}[1]{{\color{RoyalBlue}#1}} 
\newcommand{\demph}[1]{\dempfcolor{{\sl #1}}}

% comment exactly one of the following line to show / hide the solutions
% \newcommand{\solution}[1]{{\purple #1}} % uncomment to show the solutions
\newcommand{\solution}[1]{} % uncomment to hide the solutions


\newenvironment{augmentedmatrix}[1] % environment for making augmented matrix
{\left[\begin{array}{#1}}
    {\end{array}\right]}


\title{Lecture Notes for Math 372: \\Elementary Probability and Statistics}
\date{Last updated: \today}
% \author{mh}

\begin{document}
\begin{center}
  \section*{Math 307: Practice Midterm 2}
  \textit{Date of exam: November 24, in class}
\end{center}

\textit{Instructions: You have 50 minutes. Calculators and notes are not
  allowed. This practice exam is slightly longer than the actual midterm.}

\begin{problem}
  State precise definitions of (a) ``linear independence'', (b) ``span'' and (c) ``basis''.
\end{problem}

\begin{problem}
  Determine whether the following statements are true or false. You don't need
  to show your work or justify your answer.
  \begin{enumerate}[(a)]
    \item The matrix $
    \begin{bmatrix}
      -1&0\\
      0&1
    \end{bmatrix}
    $
    is a rotation. %false -- practice test
    \item Let $A$ be an $n\times n$ matrix and $b\in \R^{n}$, then $AX=b$ has a unique
    solution $X$ if $\ker(A)= \left\{0\right\}$. %true
    \item If $A$ is a $4\times 6$ matrix, then its nullspace is a plane
    passing through the origin. %false - practice
    \item The set of functions $X = \left\{f:\R \to \R\mid  f(.4) = 1\right\}$
    forms a vector space. %false
    \item Any five vectors in $\R^{4}$ are linearly dependent. %true
    \item If $A$ is an $m\times n$ matrix, then the dimension of the row space
    of $A$ equals the dimension of the column space. %true
    \item If $A,B$ are $n\times n$ matrices, then $\det(A+B)=\det(A)+\det(B)$.
    % false
    \item Let $A$ be an $m\times n$ matrix, and let $T(X) = AX$ be a linear
    transformation. Then the domain of $T$ is $\R^{n}$ and the range is
    $\R^{m}$. %true
    \item An $n\times n$ matrix has rank $n$ if and only if its columns are
    linearly independent. %true
    \item Let $A,B$ be $n\times n$ matrices. If $A$ and $B$ are invertible,
    then $AB$ is also invertible.
    \item The set of $2\times2$ matrices of rank 1 is a subspace.
    \item If we know $T(v)$ for $n$ different nonzero vectors in $\R^{n}$,
    then we know $T(v)$ for every vector $v\in \R^{n}$.
    \item If a square matrix $A$ has independent columns, so does $A^{2}$.
    \item If $T:\R^{3}\to \R^{3}$ is a linear transformation, then $T^{2}$ is
    also a linear transformation.
    \item Every subspace of $\R^{4}$ is the kernel of some matrix.
    \item The vectors 
    $\begin{bmatrix}
      -1\\2\\1
    \end{bmatrix},
    \begin{bmatrix}
      1\\1\\0
    \end{bmatrix},
    \begin{bmatrix}
      0\\1\\1
    \end{bmatrix},
    \begin{bmatrix}
      1\\1\\-1
    \end{bmatrix}$ form a basis for $\R^{3}$. % true
  \end{enumerate}
\end{problem}

\begin{problem}
  Let $A =
  \begin{bmatrix}
    10&-9\\
    4&-2
  \end{bmatrix}
  $
  \begin{enumerate}[(a)]
    \item By what factor does the linear transformation of $A$ scale area?
    \item Find the eigenvalues of $A$.
    \item For each eigenvalue of $A$, find a basis for the eigenspace.
  \end{enumerate}
\end{problem}


\begin{problem}
  Find the inverse of $
  \begin{bmatrix}
    1&1&1\\
    1&2&-1\\
    1&0&2
  \end{bmatrix}
  $
\end{problem}


\begin{problem}
  Let
  \begin{equation*}
    A =
    \begin{bmatrix}
      1&0&-1&3\\
      2&1&4&-1\\
      4&1&2&5
    \end{bmatrix}
  \end{equation*}
  Find bases for
  \begin{enumerate}[(a)]
    \item Find a basis for $RS(A)$
    \item Find a basis for $NS(A)$
    \item Find a basis for $CS(A)$
    \item What is the rank of $A$?
  \end{enumerate}
\end{problem}


\begin{problem}
  The planes given by the equations
  \begin{equation*}
    x+3y+3z = 1 \quad \text{and} \quad  x-y+z=1
  \end{equation*}
  intersect to form a line. Find an equation of the line and write your answer
  in the form
  \begin{equation*}
    \begin{bmatrix}
      x_{1}\\x_{2}\\x_{3}
    \end{bmatrix}
    =
    \begin{bmatrix}
      c_{1}\\c_{2}\\c_{3}
    \end{bmatrix}
    +
    \begin{bmatrix}
      d_{1}\\d_{2}\\d_{3}
    \end{bmatrix}
    x_{3}
  \end{equation*}
  where $x_{3}$ is a free variable, and $c_{1},c_{2},c_{3},d_{1},d_{2},d_{3}$
  are scalars.
\end{problem}


\begin{problem}
  Let
  \begin{equation*}
    A =
    \begin{bmatrix}
      3&-2\\
      1&2
    \end{bmatrix}
    \quad
    B =
    \begin{bmatrix}
      1&-4\\
      -2&3
    \end{bmatrix}.
  \end{equation*}
  \begin{enumerate}[(a)]
    \item Find the determinants of  $A$ and $B$.
    \item Find $\det(AB)$, $\det(A^{-1})$, and $\det(B^{\top}A^{-1})$.
    \item Show that $\det(A+B)$ is not the same as $\det(A)+\det(B)$.
  \end{enumerate}
\end{problem}


\begin{problem}
  Compute the determinant of
  \begin{equation*}
    A =
    \begin{bmatrix}
      6&-5&1&3\\
      3&1&-2&-1\\
      0&10&0&0\\
      3&3&0&3
    \end{bmatrix}
  \end{equation*}
\end{problem}

\begin{problem}
  Solve the system of equations using row reduction
  \begin{align*}
    2x+3y-4z&=3\\
    4x+6y-4z&=6\\
    8x+12y-4z&=14
  \end{align*}
\end{problem}


\begin{problem}
  Compute a basis for the nullspace of the following matrix:
  \begin{equation*}
    \begin{augmentedmatrix}{ccccc|c}
      1&0&2&3&4&0\\
      0&1&5&6&7&0\\
      0&1&5&6&7&0\\
      0&0&0&0&0&0\\
    \end{augmentedmatrix}
  \end{equation*}
\end{problem}


\end{document}