\documentclass[10pt]{article}
% Math Packages
\usepackage{amsmath, mathtools}
\usepackage{amssymb}
\usepackage{amsthm}
\usepackage{amsfonts}
\usepackage{bbm}
\usepackage{breqn}
\usepackage[margin=1in]{geometry}
\usepackage{graphicx}
\usepackage{tikz}
\usetikzlibrary{arrows.meta}
\usetikzlibrary{calc}
\usepackage{forest}
\usepackage{tikz-qtree}
\graphicspath{ {./images/} }
\usepackage{hyperref}
\usepackage[capitalize]{cleveref}
\usepackage[shortlabels]{enumitem}
\usetikzlibrary{arrows,matrix,positioning}
\usepackage{multicol}

% for the pipe symbol
\usepackage[T1]{fontenc}

% Citing theorems by name. (source: https://tex.stackexchange.com/questions/109843/cleveref-and-named-theorems)
\makeatletter
\newcommand{\ncref}[1]{\cref{#1}\mynameref{#1}{\csname r@#1\endcsname}}

\def\mynameref#1#2{%
  \begingroup
  \edef\@mytxt{#2}%
  \edef\@mytst{\expandafter\@thirdoffive\@mytxt}%
  \ifx\@mytst\empty\else
  \space(\nameref{#1})\fi
  \endgroup
}
\makeatother

% Colorful Notes
\usepackage{color} \definecolor{Red}{rgb}{1,0,0} \definecolor{Blue}{rgb}{0,0,1}
\definecolor{Purple}{rgb}{.5,0,.5} \def\red{\color{Red}} \def\blue{\color{Blue}}
\def\gray{\color{gray}} \def\purple{\color{Purple}}
\newcommand{\rnote}[1]{{\red [#1]}} % \rnote{foo} gives '[foo]' in red
\newcommand{\pnote}[1]{{\purple [#1]}} % \pnote{foo} gives '[foo]' in purple
\newcommand{\bnote}[1]{{\blue #1}} % \bnote{foo} gives 'foo' in blue
\newcommand{\gnote}[1]{{\gray #1}} % \gnote{foo} gives 'foo' in gray
\newcommand{\Max}[1]{{\purple [#1]}} % \bnote{foo} then 'foo' is blue


% Claim numbering (the counter restarts after each proof environment)
\newcounter{claimcount}
\setcounter{claimcount}{0}
\newenvironment{claim}{\refstepcounter{claimcount}\par\addvspace{\medskipamount}\noindent\textbf{Claim \arabic{claimcount}:}}{}
\usepackage{etoolbox}
\AtBeginEnvironment{proof}{\setcounter{claimcount}{0}}
\newenvironment{claimproof}{\par\addvspace{\medskipamount}\noindent\textit{Proof of Claim  \arabic{claimcount}.}}{\hfill\ensuremath{\qedsymbol} \tiny{Claim}

  \medskip}
% Add claim support to cleverref
\crefname{claimcount}{Claim}{Claims}


% Math Environments
\newtheorem{theorem}{Theorem}
\newtheorem{assumption}[theorem]{Assumption}
\newtheorem{lemma}[theorem]{Lemma}
\newtheorem{proposition}[theorem]{Proposition}
\newtheorem{corollary}[theorem]{Corollary}
\newtheorem{question}[theorem]{Question}
\theoremstyle{definition}
\newtheorem{definition}[theorem]{Definition}
\newtheorem{remark}[theorem]{Remark}
\newtheorem{example}[theorem]{Example}
\newtheorem{notation}[theorem]{Notation}
\newtheorem{problem}[theorem]{Problem}

% Matrices and Column Vectors. 
\usepackage{stackengine}
\setstackgap{L}{1.0\normalbaselineskip}
\usepackage{tabstackengine}
\setstackEOL{;}% row separator
\setstackTAB{,}% column separator
\setstacktabbedgap{1ex}% inter-column gap
\setstackgap{L}{1.5\normalbaselineskip}% inter-row baselineskip
\let\nmatrix\bracketMatrixstack  %Usage: \nmatrix{1,2,3\4,5,6}
\newcommand\cv[1]{\setstackEOL{,}\bracketMatrixstack{#1}} %usage: \cv{1,2,3}

% Custom Math Coqmmands
\newcommand{\vt}{\vskip 5mm} % vertical space
\newcommand{\fl}{\noindent\textbf} % first line
\newcommand{\Fl}{\vt\noindent\textbf} % first line with space above
\newcommand{\norm}[1]{\left\lVert#1\right\rVert} % norm
\newcommand{\pnorm}[1]{\left\lVert#1\right\rVert_p} % p-norm
\newcommand{\qnorm}[1]{\left\lVert#1\right\rVert_q} % q-norm
\newcommand{\1}[1]{\textbf{1}_{\left[#1\right]}} % indicator function 
\def\limn{\lim_{n\to\infty}} % shortcut for lim as n-> infinity
\def\sumn{\sum_{n=1}^{\infty}} % shortcut for sum from n=1 to infinity
\def\sumkn{\sum_{k=1}^{n}} % shortcut for sum from k=1 to n
\def\sumin{\sum_{i=1}^{n}} % shortcut for sum from i=1 to n
\def\SAs{\sigma\text{-algebras}} % shortcut for $\sigma$-algebras
\def\SA{\sigma\text{-algebra}} % shortcut for $\sigma$-algebra
\def\Ft{\mathcal{F}_t} % time-indexed sigma-algebra (t)
\def\Fs{\mathcal{F}_s} % time-indexed sigma-algebra (s)
\def\F{\mathcal{F}} % sigma-algebra
\def\G{\mathcal{G}} % sigma-algebra
\def\R{\mathbb{R}} % Real numbers
\def\N{\mathbb{N}} % Natural numbers
\def\Z{\mathbb{Z}} % Integers
\def\E{\mathbb{E}} % Expectation
\def\P{\mathbb{P}} % Probability
\def\Q{\mathbb{Q}} % Q probability
\def\dist{\text{dist}} %Text 'dist' for things like 'dist(x,y)'
\newcommand{\indep}{\perp \!\!\! \perp}  %independence symbol
\def\Var{\mathrm{Var}} % Variance
\def\tr{\mathrm{tr}} % trace

% Brackets and Parentheses
\def\[{\left [}
    \def\]{\right ]}
% \def\({\left (}
%   \def\){\right )}



\usepackage{color}
\definecolor{Red}{rgb}{1,0,0}
\definecolor{Blue}{rgb}{0,0,1}
\definecolor{Purple}{rgb}{.5,0,.5}
\def\red{\color{Red}}
\def\blue{\color{Blue}}
\def\gray{\color{gray}}
\def\purple{\color{Purple}}
\definecolor{RoyalBlue}{cmyk}{1, 0.50, 0, 0}
\newcommand{\dempfcolor}[1]{{\color{RoyalBlue}#1}} 
\newcommand{\demph}[1]{\dempfcolor{{\sl #1}}}

% comment exactly one of the following line to show / hide the solutions
% \newcommand{\solution}[1]{{\purple #1}} % uncomment to show the solutions
\newcommand{\solution}[1]{} % uncomment to hide the solutions


\newenvironment{augmentedmatrix}[1] % environment for making augmented matrix
{\left[\begin{array}{#1}}
    {\end{array}\right]}


\title{Lecture Notes for Math 372: \\Elementary Probability and Statistics}
\date{Last updated: \today}
% \author{mh}

\begin{document}
\begin{center}
  \section*{Math 307: Practice Midterm 1}
  \textit{Date of exam: November 24, in class}
\end{center}

\textit{Instructions: You have 50 minutes. Calculators and notes are not
  allowed. This practice exam is slightly longer than the actual midterm.}

\begin{problem}
  State precise definitions of the following terms:
  \begin{enumerate}[(a)]
    \item null space of $A$
    \item  ``eigenvector'' and ``eigenvalue'' of a matrix $A$
  \end{enumerate}
\end{problem}

\begin{problem}
  Determine whether the following statements are true or false. You don't need
  to show your work or justify your answer.
  \begin{enumerate}[(a)]
    \item If $A$ is an $n\times n$ matrix of rank $n-1$, then its nullspace is a
    line through the origin. %true - practice
    \item The matrix $
    \begin{bmatrix}
      1&0\\
      0&-1
    \end{bmatrix}
    $ is a rotation %false -- test
    \item The vectors
    $\begin{bmatrix}
      1\\1\\2
    \end{bmatrix},
    \begin{bmatrix}
      1\\2\\3
    \end{bmatrix},
    \begin{bmatrix}
      2\\3\\5
    \end{bmatrix}$
    span a dimension 3 subspace of $\R^{4}$. %false
    \item If $A$ is a $5\times 8$ matrix with $\dim(CS(A))=3$, then
    $\dim(NS(A))=2$. %false
    \item Let $A$ and $B$ be $n\times n$ matrices. If $T(v)=Av$ and $S(v)=Bv$ 
    for all $v\in \R^{n}$, then $T\circ S(v) = ABv$. %true
    \item The transformation $T(x)=x+11$ is linear. %false
    \item The matrix $A^{\top}A$ is symmetric. %true
    \item The columns of an invertible $n\times n$ matrix form a basis of
    $\R^{n}$. %true
    \item If the columns of $A$ are linearly independendent, then $Ax=b$ has
    exactly one solution for every $b$.
    \item If $v$ is an eigenvector of $A$, then it is an eigenvector of
    $A^{2}$ %true
    \item Let $A$ be an $m\times n$ matrix, and let $T(X) = AX$ be a linear
    transformation. Then the domain of $T$ is $\R^{n}$ and the range is
    $\R^{m}$. %true
    \item If $A,B$ are $n\times n$ matrices, then $(A+B)^{2}=A^{2}+2AB+B^{2}$.
    % false
    \item Suppose the only solution to $Ax=0$ ($m$ equations, $n$ unknowns) is
    $x=0$. Then $A$ has rank $n$.
    \item Let $A$ be an $n\times n$ matrix. If $Ax=0$ for some nonzero vector
    $x$, then the equation $Ax=0$ has infinitely many solutions.
    \item The eigenvalues of a projection matrix are always 0 or 1.
  \end{enumerate}
\end{problem}




\begin{problem} % practice exam
  Let $
  A = \begin{bmatrix}
    1&1&0&3\\
    1&1&1&-2\\
    3&3&2&-1
  \end{bmatrix}
  $
  \begin{enumerate}[(a)]
    \item Find a basis for $CS(A)$.
    \item Find a basis for $RS(A)$.
    \item Find a basis for $NS(A)$.
    \item What is the rank of $A$?
    \item Is $A$ invertible?
  \end{enumerate}
\end{problem}


\begin{problem} % practice exam
  \textit{[Note: there will be a problem like this on the exam.]}
  Let $T:\R^{2}\to \R^{2}$ be the projection onto the line $y=-x$. Let
  $\alpha=\left\{e_{1},e_{2}\right\}$
  \begin{enumerate}[(a)]
    \item Find two linearly independent eigenvectors $\beta_{1},\beta_{2}$ of
    $T$ and sketch them.
    \item What is $[T]_{\beta}^{\beta}$?
    \item Let $P$ be the change-of-basis matrix from $\alpha$ to $\beta$.
    Find $P$ and $P^{-1}$.
    \item Use your answer to the previous part to find
    $[T]_{\alpha}^{\alpha}$.
    \item What are the eigenvalues of $T$?
    \item Verify that $[T]_{\alpha}^{\alpha}$ and $[T]_{\beta}^{\beta}$ have
    the same characteristic polynomial.
  \end{enumerate}
\end{problem}

\begin{problem}
  Let $A =
  \begin{bmatrix}
    6&-8\\
    4&-6
  \end{bmatrix}
  $
  \begin{enumerate}[(a)]
    \item By what factor does the linear transformation of $A$ scale area?
    \item Find the eigenvalues of $A$.
    \item For each eigenvalue of $A$, find a basis for the eigenspace.
  \end{enumerate}
\end{problem}



\begin{problem}
  Let $A=
  \begin{bmatrix}
    1&0\\
    2&1
  \end{bmatrix}$.
  \begin{enumerate}[(a)]
    \item Sketch the image of the unit square $\left\{(x,y): 0 \leq x \leq 1 \text{
        and }0 \leq y \leq 1\right\}$ under the transformation of the matrix 
    \item Compute the eigenvalues of $A$, and and eigenbasis for each
    eigenvalue.
    \item What is the inverse of $A$?
  \end{enumerate}
\end{problem}


\begin{problem} %Practice exam
  Show that $
  \begin{bmatrix}
    1\\3\\-1
  \end{bmatrix},
  \begin{bmatrix}       
    0\\-1\\2
  \end{bmatrix},
  \begin{bmatrix}
    2\\1\\3
  \end{bmatrix}
  $
  form a basis for $\R^{3}$.
\end{problem}



\begin{problem}
  Solve the following linear system:
  \begin{align*}
    2x+3y&=6\\
    2x+y&=2\\
    x-y&=-1
  \end{align*}
  Provide a sketch and interpret your results geometrically.
\end{problem}


\begin{problem}
  Suppose $T$ is a linear transformation such that $T
  \begin{bmatrix}
    1\\1
  \end{bmatrix}
  =
  \begin{bmatrix}
    6\\6
  \end{bmatrix}
  $ and $T
  \begin{bmatrix}
    2\\0
  \end{bmatrix}
  =
  \begin{bmatrix}
    4\\0
  \end{bmatrix}
  $. 
  \begin{enumerate}[(a)]
    \item Find $
    T\begin{bmatrix}
      2\\2
    \end{bmatrix}
    $
    \item Find $
    T\begin{bmatrix}
      5\\1
    \end{bmatrix} $
    \item Find   $
    T\begin{bmatrix}
      0\\1
    \end{bmatrix}
    $
    \item Find $T
    \begin{bmatrix}
      a\\b
    \end{bmatrix}
    $ where $a,b$ are arbitrary real numbers.
    \item Write down a $2\times 2$ matrix that gives the transformation $T$.
  \end{enumerate}
\end{problem}

\end{document}