\documentclass[10pt]{article}
% Math Packages
\usepackage{amsmath, mathtools}
\usepackage{amssymb}
\usepackage{amsthm}
\usepackage{amsfonts}
\usepackage{bbm}
\usepackage{breqn}
\usepackage[margin=1in]{geometry}
\usepackage{graphicx}
\usepackage{tikz}
\usetikzlibrary{arrows.meta}
\usetikzlibrary{calc}
\usepackage{forest}
\usepackage{tikz-qtree}
\graphicspath{ {./images/} }
\usepackage{hyperref}
\usepackage[capitalize]{cleveref}
\usepackage[shortlabels]{enumitem}
\usetikzlibrary{arrows,matrix,positioning}
\usepackage{multicol}

% for the pipe symbol
\usepackage[T1]{fontenc}

% Citing theorems by name. (source: https://tex.stackexchange.com/questions/109843/cleveref-and-named-theorems)
\makeatletter
\newcommand{\ncref}[1]{\cref{#1}\mynameref{#1}{\csname r@#1\endcsname}}

\def\mynameref#1#2{%
  \begingroup
  \edef\@mytxt{#2}%
  \edef\@mytst{\expandafter\@thirdoffive\@mytxt}%
  \ifx\@mytst\empty\else
  \space(\nameref{#1})\fi
  \endgroup
}
\makeatother

% Colorful Notes
\usepackage{color} \definecolor{Red}{rgb}{1,0,0} \definecolor{Blue}{rgb}{0,0,1}
\definecolor{Purple}{rgb}{.5,0,.5} \def\red{\color{Red}} \def\blue{\color{Blue}}
\def\gray{\color{gray}} \def\purple{\color{Purple}}
\newcommand{\rnote}[1]{{\red [#1]}} % \rnote{foo} gives '[foo]' in red
\newcommand{\pnote}[1]{{\purple [#1]}} % \pnote{foo} gives '[foo]' in purple
\newcommand{\bnote}[1]{{\blue #1}} % \bnote{foo} gives 'foo' in blue
\newcommand{\gnote}[1]{{\gray #1}} % \gnote{foo} gives 'foo' in gray
\newcommand{\Max}[1]{{\purple [#1]}} % \bnote{foo} then 'foo' is blue


% Claim numbering (the counter restarts after each proof environment)
\newcounter{claimcount}
\setcounter{claimcount}{0}
\newenvironment{claim}{\refstepcounter{claimcount}\par\addvspace{\medskipamount}\noindent\textbf{Claim \arabic{claimcount}:}}{}
\usepackage{etoolbox}
\AtBeginEnvironment{proof}{\setcounter{claimcount}{0}}
\newenvironment{claimproof}{\par\addvspace{\medskipamount}\noindent\textit{Proof of Claim  \arabic{claimcount}.}}{\hfill\ensuremath{\qedsymbol} \tiny{Claim}

  \medskip}
% Add claim support to cleverref
\crefname{claimcount}{Claim}{Claims}


% Math Environments
\newtheorem{theorem}{Theorem}
\newtheorem{assumption}[theorem]{Assumption}
\newtheorem{lemma}[theorem]{Lemma}
\newtheorem{proposition}[theorem]{Proposition}
\newtheorem{corollary}[theorem]{Corollary}
\newtheorem{question}[theorem]{Question}
\theoremstyle{definition}
\newtheorem{definition}[theorem]{Definition}
\newtheorem{remark}[theorem]{Remark}
\newtheorem{example}[theorem]{Example}
\newtheorem{notation}[theorem]{Notation}
\newtheorem{problem}[theorem]{Problem}

% Matrices and Column Vectors. 
\usepackage{stackengine}
\setstackgap{L}{1.0\normalbaselineskip}
\usepackage{tabstackengine}
\setstackEOL{;}% row separator
\setstackTAB{,}% column separator
\setstacktabbedgap{1ex}% inter-column gap
\setstackgap{L}{1.5\normalbaselineskip}% inter-row baselineskip
\let\nmatrix\bracketMatrixstack  %Usage: \nmatrix{1,2,3\4,5,6}
\newcommand\cv[1]{\setstackEOL{,}\bracketMatrixstack{#1}} %usage: \cv{1,2,3}

% Custom Math Coqmmands
\newcommand{\vt}{\vskip 5mm} % vertical space
\newcommand{\fl}{\noindent\textbf} % first line
\newcommand{\Fl}{\vt\noindent\textbf} % first line with space above
\newcommand{\norm}[1]{\left\lVert#1\right\rVert} % norm
\newcommand{\pnorm}[1]{\left\lVert#1\right\rVert_p} % p-norm
\newcommand{\qnorm}[1]{\left\lVert#1\right\rVert_q} % q-norm
\newcommand{\1}[1]{\textbf{1}_{\left[#1\right]}} % indicator function 
\def\limn{\lim_{n\to\infty}} % shortcut for lim as n-> infinity
\def\sumn{\sum_{n=1}^{\infty}} % shortcut for sum from n=1 to infinity
\def\sumkn{\sum_{k=1}^{n}} % shortcut for sum from k=1 to n
\def\sumin{\sum_{i=1}^{n}} % shortcut for sum from i=1 to n
\def\SAs{\sigma\text{-algebras}} % shortcut for $\sigma$-algebras
\def\SA{\sigma\text{-algebra}} % shortcut for $\sigma$-algebra
\def\Ft{\mathcal{F}_t} % time-indexed sigma-algebra (t)
\def\Fs{\mathcal{F}_s} % time-indexed sigma-algebra (s)
\def\F{\mathcal{F}} % sigma-algebra
\def\G{\mathcal{G}} % sigma-algebra
\def\R{\mathbb{R}} % Real numbers
\def\N{\mathbb{N}} % Natural numbers
\def\Z{\mathbb{Z}} % Integers
\def\E{\mathbb{E}} % Expectation
\def\P{\mathbb{P}} % Probability
\def\Q{\mathbb{Q}} % Q probability
\def\dist{\text{dist}} %Text 'dist' for things like 'dist(x,y)'
\newcommand{\indep}{\perp \!\!\! \perp}  %independence symbol
\def\Var{\mathrm{Var}} % Variance
\def\tr{\mathrm{tr}} % trace

% Brackets and Parentheses
\def\[{\left [}
    \def\]{\right ]}
% \def\({\left (}
%   \def\){\right )}



\usepackage{color}
\definecolor{Red}{rgb}{1,0,0}
\definecolor{Blue}{rgb}{0,0,1}
\definecolor{Purple}{rgb}{.5,0,.5}
\def\red{\color{Red}}
\def\blue{\color{Blue}}
\def\gray{\color{gray}}
\def\purple{\color{Purple}}
\definecolor{RoyalBlue}{cmyk}{1, 0.50, 0, 0}
\newcommand{\dempfcolor}[1]{{\color{RoyalBlue}#1}} 
\newcommand{\demph}[1]{\dempfcolor{{\sl #1}}}

% comment exactly one of the following line to show / hide the solutions
% \newcommand{\solution}[1]{{\purple #1}} % uncomment to show the solutions
\newcommand{\solution}[1]{} % uncomment to hide the solutions


\newenvironment{augmentedmatrix}[1] % environment for making augmented matrix
{\left[\begin{array}{#1}}
    {\end{array}\right]}


\title{Additional practice problems for the final exam}
\date{Last updated: \today}
% \author{mh}

\begin{document}
\begin{center}
  \section*{Additional practice problems for the final exam}
\end{center}

\begin{problem}[change of basis - similar to updated practice exam problem]
  Let $T:\R^{2}\to \R^{2}$ be the linear transformation defined geometrically
  by first \textbf{projecting} space onto the line $y=2x$, and then reflecting
  across the line $y=-\frac{1}{2}x$.
  \begin{enumerate}[(a),itemsep=.1em]
    \item Find two linearly independent eigenvectors $\beta_{1}$ and
    $\beta_{2}$. Sketch them.

    \textit{(Hint: the line $y=-\frac{1}{2}x$ is perpendicular to $y=2x$)}

    \item Find eigenvalues $\lambda_{1}$ and $\lambda_{2}$ corresponding to
    $\beta_{1}$ and $\beta_{2}$.

    \item Let $\beta=\left\{\beta_{1},\beta_{2}\right\}$. Find
    $[T(\beta_{1})]_{\beta}$ and $[T(\beta_{2})]_{\beta}$.
    \item \label{item:1} Find $[T]_{\beta}^{\beta}$.
    \item \label{item:5} Let $\alpha=\left\{e_{1},e_{2}\right\}$ where $e_{1},e_{2}$ are the
    standard basis vectors. Find the change of basis matrix $P$ from $\alpha$
    to $\beta$, and also find $P^{-1}$.
    \item \label{item:4} Use your answers to parts \ref{item:1} and
    \ref{item:5} to find $[T]_{\alpha}^{\alpha}$.
    \item Find $[T(e_{1})]_{\alpha}$ and $[T(e_{2})]_{\alpha}$. (Hint: use
    your answer to part \ref{item:4}).
    \item Verify  $[T]_{\alpha}^{\alpha}$ and $[T]_{\beta}^{\beta}$ have
    the same characteristic polynomial.
  \end{enumerate}
\end{problem}

Here are two more problems in this vein:

\begin{problem}[change of basis]
  Let $T:\R^{2}\to \R^{2}$ be the \textbf{projection} onto the line $y=2x$.
  \begin{enumerate}[(a)]
    \item Find two linearly independent eigenvectors $\beta_{1},\beta_{2}$ of
    $T$ and sketch them.
    \item What are the eigenvalues $\lambda_{1}$ and $\lambda_{2}$
    corresponding to $\beta_{1}$ and $\beta_{2}$?
    \item What is $[T]_{\beta}^{\beta}$?
    \item Let $\alpha=\left\{e_{1},e_{2}\right\}$, and let $P$ be the
    change-of-basis matrix from $\alpha$ to $\beta$. Find $P$ and $P^{-1}$.
    \item Use your answer to the previous part to find
    $[T]_{\alpha}^{\alpha}$. Where does $T$ send $e_{1}$ and $e_{2}$?
    \item Verify that $[T]_{\alpha}^{\alpha}$ and $[T]_{\beta}^{\beta}$ have
    the same characteristic polynomial.
  \end{enumerate}
\end{problem}

\begin{problem}[change of basis]
  Let $T:\R^{2}\to \R^{2}$ be the linear tranformation defined by first
  \textbf{reflecting} across the line $y=x$ and then \textbf{dilating} space by a factor of $4$.
  \begin{enumerate}[(a)]
    \item Find two linearly independent eigenvectors $\beta_{1},\beta_{2}$ of
    $T$ and sketch them.
    \item What are the eigenvalues $\lambda_{1}$ and $\lambda_{2}$
    corresponding to $\beta_{1}$ and $\beta_{2}$?
    \item What is $[T]_{\beta}^{\beta}$?
    \item Let $\alpha=\left\{e_{1},e_{2}\right\}$, and let $P$ be the
    change-of-basis matrix from $\alpha$ to $\beta$. Find $P$ and $P^{-1}$.
    \item Use your answer to the previous part to find
    $[T]_{\alpha}^{\alpha}$. Where does $T$ send $e_{1}$ and $e_{2}$?
    \item Verify that $[T]_{\alpha}^{\alpha}$ and $[T]_{\beta}^{\beta}$ have
    the same characteristic polynomial.
  \end{enumerate}
\end{problem}



\begin{problem}[differential equation]
  Let $A =
  \begin{bmatrix}
    3&0\\
    8&-1
  \end{bmatrix}
  $
  \begin{enumerate}[(a)]
    \item Let $Y(t) =
    \begin{bmatrix}
      y_{1}(t)\\y_{2}(t)
    \end{bmatrix}
    $. Find the general solution of the differential equation $Y'=AY$.
    \item Solve the initial value problem
    \begin{equation*}
      \left\{ \begin{array}{l@{}l} Y'=AY \\ Y(0) =
          \begin{bmatrix}
            1\\1
          \end{bmatrix}
        \end{array}\right.
    \end{equation*}
  \end{enumerate}
\end{problem}



\begin{problem}[differential equation]
  Find the general solution to $Y'=AY$ when $A=
  \begin{bmatrix}
    4&0&1\\
    -2&1&0\\
    -2&0&1
  \end{bmatrix}$.
\end{problem}



\begin{problem}[determinants]
  Let
  \begin{equation*}
    A =
    \begin{bmatrix}
      3&-2\\
      1&2
    \end{bmatrix}
    \quad
    B =
    \begin{bmatrix}
      1&-4\\
      -2&3
    \end{bmatrix}.
  \end{equation*}
  \begin{enumerate}[(a)]
    \item Find the determinants of  $A$ and $B$.
    \item Find $\det(AB)$, $\det(A^{-1})$, and $\det(B^{\top}A^{-1})$.
    \item Show that $\det(A+B)$ is not the same as $\det(A)+\det(B)$.
    \item \label{item:4} Diagonalize $B$ by writing it as $B=P\Lambda P^{-1}$,
    for some diagonal matrix $\Lambda$. (For full credit, you need to find
    $P, P^{-1}$ and $\Lambda$.)

    % (Hint: the inverse of a $2\times 2$ matrix $
    % \begin{bmatrix}
    %   a&b\\c&d
    % \end{bmatrix}
    % $  is $\frac{1}{ad-bc}
    % \begin{bmatrix}
    %   d&-b\\
    %   -c&a
    % \end{bmatrix}
    % $.)
  \end{enumerate}
\end{problem}



\begin{problem}[composition]
  Consider the two linear transformations $T: \R^{3} \to \R^{4}$ and $S:
  \R^{4}\to \R^{2}$ defined by
  \begin{equation*}
    T
    \begin{bmatrix}
      x_{1}\\x_{2}\\x_{3}
    \end{bmatrix}
    =
    \begin{bmatrix}
      x_{1}\\
      x_{1}+x_{2}\\
      x_{2}+x_{3}\\
      x_{3}
    \end{bmatrix}
    \quad \text{and} \quad
    S
    \begin{bmatrix}
      x_{1}\\x_{2}\\x_{3}\\x_{4}
    \end{bmatrix}
    =
    \begin{bmatrix}
      2x_{1}-x_{2}\\
      x_{3}-2x_{4}
    \end{bmatrix}
  \end{equation*}
  \begin{enumerate}[(a)]
    \item Write the standard matrix for $T$
    \item Write the standard matrix for $S$
    \item Write the standard matrix for $S\circ T $
  \end{enumerate}
\end{problem}


\begin{problem}[system with variables] %[Practice test]
  Let $a,b,c,d\in \R$. Solve the following system of linear equations.
  \begin{align*}
    x+2y-z&=a\\
    x+y-2z&=b\\
    2x+y-3z&=c
  \end{align*}
  Your answer should be in terms of $a,b,$ and $c$.
\end{problem}



\begin{problem}[system with variables] %[10 points] % this was the midterm problem
  Determine the conditions on $a,b,c\in \R$ such that the following linear
  system has at least one solution:
  \begin{align*}
    x+2y-z&=a\\
    x+y-2z&=b\\
    2x+2y-4z&=c
  \end{align*}
\end{problem}


\begin{problem}[diag onalization]
  Let $A$ be a $2\times 2$ matrix such that
  \begin{itemize}
    \item $
    \begin{bmatrix}
      1\\1
    \end{bmatrix}
    $ is eigenvector with eigenvalue $2$
    \item $
    \begin{bmatrix}
      2\\3
    \end{bmatrix}
    $ is an eigenvector with eigenvalue $3$
  \end{itemize}
  \begin{enumerate}[(a)]
    \item Find $A^{3}
    \begin{bmatrix}
      1\\1
    \end{bmatrix}
    $, $A^{3}
    \begin{bmatrix}
      2\\3
    \end{bmatrix}
    $ and $A^{3}
    \begin{bmatrix}
      3\\4
    \end{bmatrix}
    $ 
    \item Find $A$
    \item Find $A^{3}$
  \end{enumerate}
\end{problem}


\begin{problem}[diagonalization]
  Consider the matrix $A =
  \begin{bmatrix}
    1&2\\
    1&0
  \end{bmatrix}
  $
  \begin{enumerate}[(a)]
    \item Diagonalize $A$. That is, find matrices $P,D$ and $P^{-1}$ such that
    $A= PDP^{-1}$.
    \item Use your diagonalization to find $A^{5}$.
  \end{enumerate}
\end{problem}

\begin{problem}[diagonalization]
  Let $A =
  \begin{bmatrix}
    2&0&0&0\\
    0&2&0&0\\
    0&0&3&1\\
    0&0&4&0
  \end{bmatrix}
  $. This matrix is diagonalizable; that is, there exists a diagonal matrix
  $D$ and an invertible matrix $P$ such that $A =PAP^{-1}$. Find $D$ and $P$.
  Do not compute $P^{-1}$.
\end{problem}






\begin{problem}[basis]
  Let $\beta = \left\{
    \begin{bmatrix}
      2\\-1
    \end{bmatrix},
    \begin{bmatrix}
      3\\4
    \end{bmatrix}
  \right\}$ be a basis for $\R^{2}$. If $[v]_{\beta}=
  \begin{bmatrix}
    -3\\2
  \end{bmatrix}
  $,  find $v$.
\end{problem}




\begin{problem}[Geometry]
  \begin{enumerate}[(a)]
    \item[]
    \item Find the matrix of the linear transformation $T:\R^{2}\to \R^{2}$ that
    rotates space by $90^{\circ}$ counterclockwise. 
    \item Find the matrix of the linear transformation defined by $S
    \begin{bmatrix}
      x\\y
    \end{bmatrix}
    =
    \begin{bmatrix}
      2x+y\\
      3y-x\\
      x-3y
    \end{bmatrix}
    $.
    \item Circle which transformation makes sense: $T\circ S$ \quad $S\circ
    T$.
    \item Write a matrix for the transformation you circled.
  \end{enumerate}
\end{problem}




\begin{problem}[geometry]
  The linear transformation $T:\R^{2} \to \R^{2}$ can be achieved by rotating
  counterclockwise by $135^{\circ}$ and then expanding vertically by a factor
  of $4$. Find the standard matrix for $T$.
\end{problem}


\begin{problem}[computation] % easy
  Let
  \begin{equation*}
    A =
    \begin{bmatrix}
      0&0&1&2\\
      0&0&1&2\\
      1&1&1&0
    \end{bmatrix}
  \end{equation*}
  \begin{enumerate}[(a),itemsep=.1em]
    \item (2 points) Find a basis for the column space of $A$.
    \item (2 points) Find a basis for the row space of $A$.
    \item (2 points) Find a basis for the null space of $A$.
    \item (2 points) What is the rank of $A$?
    \item (1 point) Is $A$ invertible?
  \end{enumerate}
\end{problem}



\begin{problem}[computation] % easy
  Let $A =
  \begin{bmatrix}
    0&-1&1\\
    1&2&0\\
    1&1&1
  \end{bmatrix}
  $.
  \begin{enumerate}[(a)]
    \item Find the rank and nullity of $A$.
    \item Find a basis for the row space of $A$.
    \item Find a basis for the column space of $A$.
    \item Find a basis for the nullspace of $A$.
  \end{enumerate}
\end{problem}



\begin{problem}
  Solve the following system by reducing the augmented matrix to reduced
  row-echelon form:
  \begin{align*}
    x+3y+2z&=2\\ 
    x \quad\ \quad-4z&=-7\\ 
    -2x-4y+3z&=-1\ 
  \end{align*}
  % solution : x = -11, y = 5, z = -1
\end{problem}



\begin{problem}[computation] % medium
  Let $A =
  \begin{bmatrix}
    1& 3& -3& -2\\
    0& -2& 4& 2\\
    2& 4& -5& -3
  \end{bmatrix}
  $. Find a basis for the nullspace and column space of $A$.
\end{problem}


\begin{problem}[computation] % hard
  Suppose $T$ is a linear transformation such that
  \begin{equation*}
    T
    \begin{bmatrix}
      1\\1\\2
    \end{bmatrix}
    =
    \begin{bmatrix}
      7\\-1\\5
    \end{bmatrix},
    \quad
    T
    \begin{bmatrix}
      1\\-1\\1
    \end{bmatrix}
    =
    \begin{bmatrix}
      6\\-2\\5
    \end{bmatrix},
    \quad
    T
    \begin{bmatrix}
      1\\1\\1
    \end{bmatrix}
    =
    \begin{bmatrix}
      6\\2\\5
    \end{bmatrix}
  \end{equation*}
  \begin{enumerate}[(a)]
    \item Find $T
    \begin{bmatrix}
      a\\b\\c
    \end{bmatrix}
    $
    \item Write down a $3\times 3$ matrix that gives the transformation $T$.
  \end{enumerate}
\end{problem}


\begin{problem}[Subspace]
  \begin{enumerate}[(a)]
    \item[]
    \item Give a precise definition of \textbf{linear subspace}.
    \item Show that $W = \left\{
      \begin{bmatrix}
        x_{1}\\x_{2}\\x_{3}
      \end{bmatrix}\in \R^{3}: x_{1}+2x_{2}-3x_{3}=0
    \right\}$ a subspace of $\R^{3}$, and find a basis for it.
    \item Is $U = \text{Span}\left\{
      \begin{bmatrix}
        1\\1\\2
      \end{bmatrix},
      \begin{bmatrix}
        1\\2\\-1
      \end{bmatrix},
      \begin{bmatrix}
        2\\3\\1
      \end{bmatrix}
    \right\}$ a subspace of $\R^{3}$? If not, justify why not. If $U$ is a
    subspace, find a basis for it.
  \end{enumerate}
\end{problem}



\newpage


\end{document}