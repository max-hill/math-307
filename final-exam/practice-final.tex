\documentclass[10pt]{article}
% Math Packages
\usepackage{amsmath, mathtools}
\usepackage{amssymb}
\usepackage{amsthm}
\usepackage{amsfonts}
\usepackage{bbm}
\usepackage{breqn}
\usepackage[margin=1in]{geometry}
\usepackage{graphicx}
\usepackage{tikz}
\usetikzlibrary{arrows.meta}
\usetikzlibrary{calc}
\usepackage{forest}
\usepackage{tikz-qtree}
\graphicspath{ {./images/} }
\usepackage{hyperref}
\usepackage[capitalize]{cleveref}
\usepackage[shortlabels]{enumitem}
\usetikzlibrary{arrows,matrix,positioning}
\usepackage{multicol}

% for the pipe symbol
\usepackage[T1]{fontenc}

% Citing theorems by name. (source: https://tex.stackexchange.com/questions/109843/cleveref-and-named-theorems)
\makeatletter
\newcommand{\ncref}[1]{\cref{#1}\mynameref{#1}{\csname r@#1\endcsname}}

\def\mynameref#1#2{%
  \begingroup
  \edef\@mytxt{#2}%
  \edef\@mytst{\expandafter\@thirdoffive\@mytxt}%
  \ifx\@mytst\empty\else
  \space(\nameref{#1})\fi
  \endgroup
}
\makeatother

% Colorful Notes
\usepackage{color} \definecolor{Red}{rgb}{1,0,0} \definecolor{Blue}{rgb}{0,0,1}
\definecolor{Purple}{rgb}{.5,0,.5} \def\red{\color{Red}} \def\blue{\color{Blue}}
\def\gray{\color{gray}} \def\purple{\color{Purple}}
\newcommand{\rnote}[1]{{\red [#1]}} % \rnote{foo} gives '[foo]' in red
\newcommand{\pnote}[1]{{\purple [#1]}} % \pnote{foo} gives '[foo]' in purple
\newcommand{\bnote}[1]{{\blue #1}} % \bnote{foo} gives 'foo' in blue
\newcommand{\gnote}[1]{{\gray #1}} % \gnote{foo} gives 'foo' in gray
\newcommand{\Max}[1]{{\purple [#1]}} % \bnote{foo} then 'foo' is blue


% Claim numbering (the counter restarts after each proof environment)
\newcounter{claimcount}
\setcounter{claimcount}{0}
\newenvironment{claim}{\refstepcounter{claimcount}\par\addvspace{\medskipamount}\noindent\textbf{Claim \arabic{claimcount}:}}{}
\usepackage{etoolbox}
\AtBeginEnvironment{proof}{\setcounter{claimcount}{0}}
\newenvironment{claimproof}{\par\addvspace{\medskipamount}\noindent\textit{Proof of Claim  \arabic{claimcount}.}}{\hfill\ensuremath{\qedsymbol} \tiny{Claim}

  \medskip}
% Add claim support to cleverref
\crefname{claimcount}{Claim}{Claims}


% Math Environments
\newtheorem{theorem}{Theorem}
\newtheorem{assumption}[theorem]{Assumption}
\newtheorem{lemma}[theorem]{Lemma}
\newtheorem{proposition}[theorem]{Proposition}
\newtheorem{corollary}[theorem]{Corollary}
\newtheorem{question}[theorem]{Question}
\theoremstyle{definition}
\newtheorem{definition}[theorem]{Definition}
\newtheorem{remark}[theorem]{Remark}
\newtheorem{example}[theorem]{Example}
\newtheorem{notation}[theorem]{Notation}
\newtheorem{problem}[theorem]{Problem}

% Matrices and Column Vectors. 
\usepackage{stackengine}
\setstackgap{L}{1.0\normalbaselineskip}
\usepackage{tabstackengine}
\setstackEOL{;}% row separator
\setstackTAB{,}% column separator
\setstacktabbedgap{1ex}% inter-column gap
\setstackgap{L}{1.5\normalbaselineskip}% inter-row baselineskip
\let\nmatrix\bracketMatrixstack  %Usage: \nmatrix{1,2,3\4,5,6}
\newcommand\cv[1]{\setstackEOL{,}\bracketMatrixstack{#1}} %usage: \cv{1,2,3}

% Custom Math Coqmmands
\newcommand{\vt}{\vskip 5mm} % vertical space
\newcommand{\fl}{\noindent\textbf} % first line
\newcommand{\Fl}{\vt\noindent\textbf} % first line with space above
\newcommand{\norm}[1]{\left\lVert#1\right\rVert} % norm
\newcommand{\pnorm}[1]{\left\lVert#1\right\rVert_p} % p-norm
\newcommand{\qnorm}[1]{\left\lVert#1\right\rVert_q} % q-norm
\newcommand{\1}[1]{\textbf{1}_{\left[#1\right]}} % indicator function 
\def\limn{\lim_{n\to\infty}} % shortcut for lim as n-> infinity
\def\sumn{\sum_{n=1}^{\infty}} % shortcut for sum from n=1 to infinity
\def\sumkn{\sum_{k=1}^{n}} % shortcut for sum from k=1 to n
\def\sumin{\sum_{i=1}^{n}} % shortcut for sum from i=1 to n
\def\SAs{\sigma\text{-algebras}} % shortcut for $\sigma$-algebras
\def\SA{\sigma\text{-algebra}} % shortcut for $\sigma$-algebra
\def\Ft{\mathcal{F}_t} % time-indexed sigma-algebra (t)
\def\Fs{\mathcal{F}_s} % time-indexed sigma-algebra (s)
\def\F{\mathcal{F}} % sigma-algebra
\def\G{\mathcal{G}} % sigma-algebra
\def\R{\mathbb{R}} % Real numbers
\def\N{\mathbb{N}} % Natural numbers
\def\Z{\mathbb{Z}} % Integers
\def\E{\mathbb{E}} % Expectation
\def\P{\mathbb{P}} % Probability
\def\Q{\mathbb{Q}} % Q probability
\def\dist{\text{dist}} %Text 'dist' for things like 'dist(x,y)'
\newcommand{\indep}{\perp \!\!\! \perp}  %independence symbol
\def\Var{\mathrm{Var}} % Variance
\def\tr{\mathrm{tr}} % trace

% Brackets and Parentheses
\def\[{\left [}
    \def\]{\right ]}
% \def\({\left (}
%   \def\){\right )}



\usepackage{color}
\definecolor{Red}{rgb}{1,0,0}
\definecolor{Blue}{rgb}{0,0,1}
\definecolor{Purple}{rgb}{.5,0,.5}
\def\red{\color{Red}}
\def\blue{\color{Blue}}
\def\gray{\color{gray}}
\def\purple{\color{Purple}}
\definecolor{RoyalBlue}{cmyk}{1, 0.50, 0, 0}
\newcommand{\dempfcolor}[1]{{\color{RoyalBlue}#1}} 
\newcommand{\demph}[1]{\dempfcolor{{\sl #1}}}

% comment exactly one of the following line to show / hide the solutions
% \newcommand{\solution}[1]{{\purple #1}} % uncomment to show the solutions
\newcommand{\solution}[1]{} % uncomment to hide the solutions


\newenvironment{augmentedmatrix}[1] % environment for making augmented matrix
{\left[\begin{array}{#1}}
    {\end{array}\right]}


\title{Lecture Notes for Math 372: \\Elementary Probability and Statistics}
\date{Last updated: \today}
% \author{mh}

\begin{document}
\begin{center}
  \section*{Math 307: Practice final exam}
\end{center}

\textit{Instructions: You have 120 minutes. Calculators and notes are not
  allowed. There are a total of 70 points on the exam.}

\begin{problem}[Definitions]
  Your answer must be precise to recieve full credit.
  \begin{enumerate}[(a)]
    \item Let $v_{1},\ldots,v_{n}$ be vectors in a vector space $V$. Define
    $\text{Span}\left\{v_{1},\ldots,v_{n}\right\}$.
    \vspace{6cm}
    \item Define \textbf{linear independence}.
    \vspace{6cm}
    \item Let $v_{1},\ldots,v_{n}$ be vectors in a vector space $V$. The
    vectors $v_{1},\ldots,v_{n}$ are a \textbf{basis} of $V$ if...
  \end{enumerate}
\end{problem}

\newpage
\begin{problem}[dimension]
  Fill in the blanks.
  \begin{enumerate}[(a)]
    \item (2 points) If the dimension of the row space of a
    $3\times 5$ matrix is $2$, then the nullspace of $A$ is a linear subspace
    of \underline{\quad \quad} with dimension \underline{\quad \quad}.
    \item (4 points) Suppose $Ax=0$ is a system of linear equations with $10$
    equations in $5$ variables, and the only solution is $x=0$. The rank is
    \underline{\quad \quad}. The columns of $A$ are linearly \underline{\quad
      \quad \quad \quad}. The rows of $A$ are linearly \underline{\quad \quad
      \quad \quad}. The rank of $A$ is \underline{ \quad \quad}.
    \item Let $A$ is a $9\times 5$ matrix. If $Ax=0$ has 3 free variables in
    its solution set, which one of the following describes the column space of
    $A$? Circle one:
    \begin{itemize}
      \item $CS(A)$ is a $2$ dimensional subspace of $\R^{9}$
      \item $CS(A)$ is a $2$ dimensional subspace of $\R^{5}$
      \item $CS(A)$ is a $7$ dimensional subspace of $\R^{9}$
      \item $CS(A)$ is a $7$ dimensional subspace of $\R^{5}$
    \end{itemize}
  \end{enumerate}
\end{problem}


\newpage
\begin{problem}[change of basis]
  Let $T:\R^{2}\to \R^{2}$ be the \textbf{reflection} across the line $y=2x$.
  \begin{enumerate}[(a),itemsep=.1em]
    \item Find two linearly independent eigenvectors $\beta_{1}$ and
    $\beta_{2}$. Sketch them.

    \textit{(Hint: the line $y=-\frac{1}{2}x$ is perpendicular to $y=2x$)}

    \item Find eigenvalues $\lambda_{1}$ and $\lambda_{2}$ corresponding to
    $\beta_{1}$ and $\beta_{2}$.

    \item Let $\beta=\left\{\beta_{1},\beta_{2}\right\}$. Find
    $[T(\beta_{1})]_{\beta}$ and $[T(\beta_{2})]_{\beta}$.
    \item \label{item:1} Find $[T]_{\beta}^{\beta}$.
    \item \label{item:5} Let $\alpha=\left\{e_{1},e_{2}\right\}$ where $e_{1},e_{2}$ are the
    standard basis vectors. Find the change of basis matrix $P$ from $\alpha$
    to $\beta$, and also find $P^{-1}$.
    \item \label{item:4} Use your answers to parts \ref{item:1} and
    \ref{item:5} to find $[T]_{\alpha}^{\alpha}$.
    \item Find $[T(e_{1})]_{\alpha}, [T(e_{2})]_{\alpha}$. (Hint: use your
    answer to part \ref{item:4}).
    \item Verify  $[T]_{\alpha}^{\alpha}$ and $[T]_{\beta}^{\beta}$ have
    the same characteristic polynomial.
  \end{enumerate}
\end{problem}

\newpage

\begin{problem}[differential equation]
  Let \begin{equation*}
    A=
    \begin{bmatrix}
      6&-8\\4&-6
    \end{bmatrix}.
  \end{equation*}
  Solve the initial value problem
  \begin{equation*}
    \left\{ \begin{array}{l@{}l} Y'=AY \\Y(0)=
        \begin{bmatrix}
          11\\6
        \end{bmatrix}
      \end{array}\right.
  \end{equation*}
  Find the general solution to $Y'=AY$ when
\end{problem}

\newpage
\begin{problem}[True-false]
  \begin{enumerate}[(a)]
    \setlength{\itemsep}{2em}
    \item[]
    \item \textbf{TRUE \mid\  FALSE} \quad  \parbox[t]{\linewidth-3cm}{%
      If $v$ is an eigenvector $A$, then $5v$ is also an eigenvector.}
    
    \item \textbf{TRUE \mid\  FALSE} \quad \parbox[t]{\linewidth-3cm}{%
      An $4\times 5$ matrix $A$ is surjective if $\text{Rank}(A) = 4$.}

    \item \textbf{TRUE \mid\ FALSE} \quad \parbox[t]{\linewidth-3cm}{%
      If $T: \R^{n} \to \R^{m}$ is a surjective linear transformation, then it
      must be the case that $m \leq n$.}

    \item \textbf{TRUE \mid\  FALSE} \quad  \parbox[t]{\linewidth-3cm}{%
      The intersection of two subspaces of a vector space is never empty.}
    
    \item \textbf{TRUE \mid\  FALSE} \quad  \parbox[t]{\linewidth-3cm}{%
      If $v_{1},\ldots,v_{n}$ is a linearly independent set of vectors,
      then $\lambda v_{1},\lambda v_{2},\ldots,\lambda v_{n}$ are also linearly
      independent.}

    \item \textbf{TRUE \mid\  FALSE} \quad  \parbox[t]{\linewidth-3cm}{%
      If $v_{1},\ldots,v_{n}$ are linearly dependent, then at least 2 of
      the vectors are scalar multiples of each other.}

    \item \textbf{TRUE \mid\  FALSE} \quad  \parbox[t]{\linewidth-3cm}{%
      If $A$ is an $n\times n$ diagonalizable matrix, then $A$ has $n$
      distinct eigenvalues.}
    
    \item \textbf{TRUE \mid\ FALSE} \quad \parbox[t]{\linewidth-3cm}{%
      If $A$ is an $m\times n$ matrix, then $\ker(A)$ is a linear subspace of
      $\R^{n}$.}

    \item \textbf{TRUE \mid\  FALSE} \quad  \parbox[t]{\linewidth-3cm}{%
      If $\lambda=0$ is an eigenvalue of $A$, then for any $b$, the
      equation $Ax=b$ has infinitely many solutions.}
    
    \item \textbf{TRUE \mid\  FALSE} \quad  \parbox[t]{\linewidth-3cm}{%
      If $v$ is a vector in $NS(B)$, then it is in  $NS(AB)$ as well.}
    
    \item \textbf{TRUE \mid\  FALSE} \quad  \parbox[t]{\linewidth-3cm}{%
      If $v$ is a vector in $NS(B)$, then it is in  $NS(BA)$ as well.}
    
    \item \textbf{TRUE \mid\  FALSE} \quad  \parbox[t]{\linewidth-3cm}{%
      Let $A,B$ be $n\times n$ matrices. If $v$ is in the column space of
      $A$, then it is also in the column space of $AB$.}
    
    \item \textbf{TRUE \mid\  FALSE} \quad  \parbox[t]{\linewidth-3cm}{%
      A linear transformation $T: \R^{n} \to \R^{n}$ is invertible if and only
      if it is surjective.}
    
    \item \textbf{TRUE \mid\  FALSE} \quad  \parbox[t]{\linewidth-3cm}{%
      A linear transformation $T: \R^{n} \to \R^{n}$ is invertible if and only
      if it is injective.}
    
    \item \textbf{TRUE \mid\  FALSE} \quad  \parbox[t]{\linewidth-3cm}{%
      An $n\times n$ matrix is diagonalizable if and only if it has $n$
      linearly independent eigenvectors.}

  \end{enumerate}
\end{problem}





\newpage
\begin{problem}[Subspace problem]
  Determine whether the following sets are subspaces or not. If the set is not
  a subspace, give a reason why not. If the set is a subspace, find a basis
  for it.
  \begin{enumerate}[(a)]
    \setlength{\itemsep}{7em}
    \item  The set
    $\left\{ \begin{bmatrix} x_{1}\\x_{2}\\x_{3}\\x_{4} \end{bmatrix}\in
      \R^{4}: x_{1}+2x_{2}+7x_{3}-x_{4}=0 \right\}$.
    \item The set $\left\{
      \begin{bmatrix}
        x\\y
      \end{bmatrix}\in \R^{2}: x+y=0 \text{ or } x-y=0
    \right\} $
    \item The set
    $\left\{ \begin{bmatrix} x\\y \end{bmatrix}\in \R^{2}: xy=0 \right\} $
    is a linear subspace of $\R^{2}$.
    \item The vectors in $\R^{4}$ for which $x_{1} = 2x_{4}$.
    \item The vectors in $\R^{4}$ for which $x_{1}+x_{2}+x_{3}=0$ and $x_{3}+x_{4}=0$.
    \item The subspace spanned by $
    \begin{bmatrix}
      1\\1\\1\\1
    \end{bmatrix},
    \begin{bmatrix}
      1\\2\\3\\4
    \end{bmatrix},
    \begin{bmatrix}
      2\\3\\4\\5
    \end{bmatrix}
    $
  \end{enumerate}
\end{problem}




\newpage
\begin{problem}[differential equation]
  A person is standing in a large pipe drops a flask of chemicals, releasing a
  chemical gas. The tube has 2 sections, $S_{1}$ and $S_{2}$, as shown:
  \begin{center}
    \includegraphics[scale=.05]{images/diffusion-DE}
  \end{center}
  Let $Y(t) =
  \begin{bmatrix}
    y_{1}(t)\\y_{2}(t)
  \end{bmatrix}
  $, where
  \begin{align*}
    y_{1}(t)  &= \text{ concentration of chemical in section }S_{1} \text{ at time }t\\
    y_{2}(t)  &= \text{ concentration of chemical in section }S_{2} \text{ at time }t.
  \end{align*}
  Further, assume that the ends of the pipe are open, so the chemical can leak
  out the ends of the tube. (The concentration outside is always zero, since
  the gas gets immediately blown away by the wind). Assume that at each time
  $t$, the diffusion rate between adjacent areas is the difference in
  concentrations.

  \begin{enumerate}[(a)]
    \item Set up a system of differential equations $Y'=AY$ describing the
    diffusion of the gas.
    \vspace{2cm}
    \item Solve your system, assuming that the initial concentration $Y(0)=
    \begin{bmatrix}
      0\\1
    \end{bmatrix}
    $. Interpret your result in words.
  \end{enumerate}
\end{problem}



\newpage
\begin{problem}[Geometry]
  Let $M=
  \begin{bmatrix}
    a&b\\
    c&d
  \end{bmatrix}
  $. Find values for $a,b,c,d$ such that the column space of $M$ is the line
  $y=x$ and the null space is the line $y=-3x$.
\end{problem}

\newpage


\begin{problem}[computation, diagonalizability]
  Let $A =
  \begin{bmatrix}
    2&0&0\\
    -2&-3&2\\
    4&-2&1
  \end{bmatrix}
  $.
  \begin{enumerate}[(a)]
    \item Find the eigenvalues of $A$.
    \item For each eigenvalue $\lambda$,  find the eigenspace $E_{\lambda}$.
    \item Is $A$ diagonalizable?
  \end{enumerate}
\end{problem}


\newpage

\begin{problem}[system with variables]
  Consider the system of equations
  \begin{equation*}
    \left\{ \begin{array}{l@{}l}
        x_{1}+x_{2}+3x_{3}=a\\
        2x_{1}+x_{2}+4x_{3}=b\\
        3x_{1}+x_{2}+5x_{3}=c\\
      \end{array}\right.
  \end{equation*}
  where $a,b,c\in \R$.
  \begin{enumerate}[(a)]
    \item \label{item:2} For conditions on $a,b,c$ such that the system has at least one
    solution.
    \item \label{item:3} When the conditions in part \ref{item:2} are satisfied, find all
    solutions $
    \begin{bmatrix}
      x_{1}\\x_{2}\\x_{3}
    \end{bmatrix}
    $ of the system.
    \item Describe the geometric shape of your answer to part \ref{item:3}.
    What is its dimension?
  \end{enumerate}
\end{problem}


\newpage
\begin{problem}[Geometry]
  The linear transformation $T:\R^{2} \to \R^{2}$ can be achieved by 
  reflecting across the $y$-axis, and then rotating counter-clockwise by
  $45^{\circ}$. Find the standard matrix for $T$ and use it to find $T
  \begin{bmatrix}
    1\\1
  \end{bmatrix}.$
\end{problem}


\newpage
\begin{problem}[computation]
  Let $A =
  \begin{bmatrix}
    1&2&3&4\\
    2&3&4&5\\
    3&4&5&6
  \end{bmatrix}
  $.
  \begin{enumerate}[(a)]
    \item Find a basis for $\ker(A)$. What is the dimension of the kernel?
    \item Find a basis for the column space of $A$. What is its dimension?
  \end{enumerate}
\end{problem}


\end{document}