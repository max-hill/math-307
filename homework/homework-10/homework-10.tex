\documentclass[10pt]{article}
% Math Packages
\usepackage{amsmath, mathtools}
\usepackage{amssymb}
\usepackage{amsthm}
\usepackage{amsfonts}
\usepackage{bbm}
\usepackage{breqn}
\usepackage[margin=1in]{geometry}
\usepackage{graphicx}
\usepackage{tikz}
\usetikzlibrary{arrows.meta}
\usetikzlibrary{calc}
\usepackage{forest}
\usepackage{tikz-qtree}
\graphicspath{ {./images/} }
\usepackage{hyperref}
\usepackage[capitalize]{cleveref}
\usepackage[shortlabels]{enumitem}
\usetikzlibrary{arrows,matrix,positioning}
\usepackage{multicol}

% % for boxed text
% \usepackage[most]{tcolorbox}%

% % for the pipe symbol
% \usepackage[T1]{fontenc}

% Citing theorems by name. (source: https://tex.stackexchange.com/questions/109843/cleveref-and-named-theorems)
\makeatletter
\newcommand{\ncref}[1]{\cref{#1}\mynameref{#1}{\csname r@#1\endcsname}}

\def\mynameref#1#2{%
  \begingroup
  \edef\@mytxt{#2}%
  \edef\@mytst{\expandafter\@thirdoffive\@mytxt}%
  \ifx\@mytst\empty\else
  \space(\nameref{#1})\fi
  \endgroup
}
\makeatother

% Colorful Notes
\usepackage{color} \definecolor{Red}{rgb}{1,0,0} \definecolor{Blue}{rgb}{0,0,1}
\definecolor{Purple}{rgb}{.5,0,.5} \def\red{\color{Red}} \def\blue{\color{Blue}}
\def\gray{\color{gray}} \def\purple{\color{Purple}}
\newcommand{\rnote}[1]{{\red [#1]}} % \rnote{foo} gives '[foo]' in red
\newcommand{\pnote}[1]{{\purple [#1]}} % \pnote{foo} gives '[foo]' in purple
\newcommand{\bnote}[1]{{\blue #1}} % \bnote{foo} gives 'foo' in blue
\newcommand{\gnote}[1]{{\gray #1}} % \gnote{foo} gives 'foo' in gray
\newcommand{\Max}[1]{{\purple [#1]}} % \bnote{foo} then 'foo' is blue


% Claim numbering (the counter restarts after each proof environment)
\newcounter{claimcount}
\setcounter{claimcount}{0}
\newenvironment{claim}{\refstepcounter{claimcount}\par\addvspace{\medskipamount}\noindent\textbf{Claim \arabic{claimcount}:}}{}
\usepackage{etoolbox}
\AtBeginEnvironment{proof}{\setcounter{claimcount}{0}}
\newenvironment{claimproof}{\par\addvspace{\medskipamount}\noindent\textit{Proof of Claim  \arabic{claimcount}.}}{\hfill\ensuremath{\qedsymbol} \tiny{Claim}

  \medskip}
% Add claim support to cleverref
\crefname{claimcount}{Claim}{Claims}


% Math Environments
\newtheorem{theorem}{Theorem}
\newtheorem{assumption}[theorem]{Assumption}
\newtheorem{lemma}[theorem]{Lemma}
\newtheorem{proposition}[theorem]{Proposition}
\newtheorem{corollary}[theorem]{Corollary}
\newtheorem{question}[theorem]{Question}
\theoremstyle{definition}
\newtheorem{definition}[theorem]{Definition}
\newtheorem{remark}[theorem]{Remark}
\newtheorem{example}[theorem]{Example}
\newtheorem{notation}[theorem]{Notation}
\newtheorem{problem}[theorem]{Problem}

% Matrices and Column Vectors. 
\usepackage{stackengine}
\setstackgap{L}{1.0\normalbaselineskip}
\usepackage{tabstackengine}
\setstackEOL{;}% row separator
\setstackTAB{,}% column separator
\setstacktabbedgap{1ex}% inter-column gap
\setstackgap{L}{1.5\normalbaselineskip}% inter-row baselineskip
\let\nmatrix\bracketMatrixstack  %Usage: \nmatrix{1,2,3\4,5,6}
\newcommand\cv[1]{\setstackEOL{,}\bracketMatrixstack{#1}} %usage: \cv{1,2,3}

% Custom Math Coqmmands
\newcommand{\vt}{\vskip 5mm} % vertical space
\newcommand{\fl}{\noindent\textbf} % first line
\newcommand{\Fl}{\vt\noindent\textbf} % first line with space above
\newcommand{\norm}[1]{\left\lVert#1\right\rVert} % norm
\newcommand{\pnorm}[1]{\left\lVert#1\right\rVert_p} % p-norm
\newcommand{\qnorm}[1]{\left\lVert#1\right\rVert_q} % q-norm
\newcommand{\1}[1]{\textbf{1}_{\left[#1\right]}} % indicator function 
\def\limn{\lim_{n\to\infty}} % shortcut for lim as n-> infinity
\def\sumn{\sum_{n=1}^{\infty}} % shortcut for sum from n=1 to infinity
\def\sumkn{\sum_{k=1}^{n}} % shortcut for sum from k=1 to n
\def\sumin{\sum_{i=1}^{n}} % shortcut for sum from i=1 to n
\def\SAs{\sigma\text{-algebras}} % shortcut for $\sigma$-algebras
\def\SA{\sigma\text{-algebra}} % shortcut for $\sigma$-algebra
\def\Ft{\mathcal{F}_t} % time-indexed sigma-algebra (t)
\def\Fs{\mathcal{F}_s} % time-indexed sigma-algebra (s)
\def\F{\mathcal{F}} % sigma-algebra
\def\G{\mathcal{G}} % sigma-algebra
\def\R{\mathbb{R}} % Real numbers
\def\N{\mathbb{N}} % Natural numbers
\def\Z{\mathbb{Z}} % Integers
\def\E{\mathbb{E}} % Expectation
\def\P{\mathbb{P}} % Probability
\def\Q{\mathbb{Q}} % Q probability
\def\dist{\text{dist}} %Text 'dist' for things like 'dist(x,y)'
\newcommand{\indep}{\perp \!\!\! \perp}  %independence symbol
\def\Var{\mathrm{Var}} % Variance
\def\tr{\mathrm{tr}} % trace

% Brackets and Parentheses
\def\[{\left [}
    \def\]{\right ]}
% \def\({\left (}
%   \def\){\right )}



\usepackage{color}
\definecolor{Red}{rgb}{1,0,0}
\definecolor{Blue}{rgb}{0,0,1}
\definecolor{Purple}{rgb}{.5,0,.5}
\def\red{\color{Red}}
\def\blue{\color{Blue}}
\def\gray{\color{gray}}
\def\purple{\color{Purple}}
\definecolor{RoyalBlue}{cmyk}{1, 0.50, 0, 0}
\newcommand{\dempfcolor}[1]{{\color{RoyalBlue}#1}} 
\newcommand{\demph}[1]{\dempfcolor{{\sl #1}}}

% comment exactly one of the following line to show / hide the solutions
% \newcommand{\solution}[1]{{\purple #1}} % uncomment to show the solutions
\newcommand{\solution}[1]{} % uncomment to hide the solutions



\title{Lecture Notes for Math 372: \\Elementary Probability and Statistics}
\date{Last updated: \today}
% \author{mh}

\begin{document}
\begin{center}
  \section*{Math 307: Homework 10}
  \textit{Due Wednesday, December 10}
\end{center}



\begin{problem}
  Let $Y(t) =
  \begin{bmatrix}
    c_{1}e^{2t}+c_{2}e^{3t}\\
    2c_{1}e^{2t}+c_{2}e^{3t}
  \end{bmatrix}$.
  \begin{enumerate}[(a)]
    \item Show that $Y(t)$ is a solution of $Y'=
    \begin{bmatrix}
      4&-1\\
      2&1
    \end{bmatrix}Y.$ \textit{(To do this, compute the LHS and RHS independently and then
      observe that they are equal).}
    \item Use the webapp
    \url{https://homepages.bluffton.edu/~nesterd/apps/slopefields.html} (or
    the sofrware \texttt{PPLANE} at \url{http://alun.math.ncsu.edu/pplane/}),
    to plot the slope field of the system. Clicking on a coordinate in the
    plot will show the trajectory of a particle starting at that point whose
    motion is governed by the differential equation. This can also be done
    precisely using the ``initial points'' tab. Do this to plot some
    trajectories.
  \end{enumerate}
\end{problem}

\begin{problem}
  Find the general solution to $Y'=AY$ when $A=
  \begin{bmatrix}
    6&-8\\4&-6
  \end{bmatrix}$ and plot the slope field of the differential equation.
\end{problem}


\begin{problem}
  Find the general solution to $Y'=AY$ when $A=
  \begin{bmatrix}
    1&2\\3&4
  \end{bmatrix}$ and plot the slope field of the differential equation.
\end{problem}


\begin{problem}
  Let $A =
  \begin{bmatrix}
    1&0\\0&-2
  \end{bmatrix}
  $.
  \begin{enumerate}[(a)]
    \item  Determine the general solution to $Y'=AY$
    \item Solve the initial value problem $Y'=AY$ with $Y(0)=
    \begin{bmatrix}
      2\\1
    \end{bmatrix} $
    \item Plot the solution to the initial value problem using the software from
    the previous problems.
  \end{enumerate}
\end{problem}

\begin{problem}[A non-diagonalizable example]
  \label{problem:non-diagonalizable-example}
  Consider the system of differential equations
  \begin{align*}
    \left\{ \begin{array}{l@{}l}
        x'(t) &= y\\ 
        y'(t) &= -x-2y 
      \end{array}\right.
  \end{align*}
  Letting $Y(t) = \begin{bmatrix} x(t)\\y(y) \end{bmatrix}$ and $A =
  \begin{bmatrix}
    0&1\\
    -1&-2
  \end{bmatrix}
  $, we can write the
  above system of differential equations in matrix form as
  \begin{equation}\label{eq:1}
    Y' =A Y.
  \end{equation}
  \begin{enumerate}[(a)]
    \item Plot the slope field and some particle trajectories using the
    software from the previous problem.
    \item Are there any straight-line trajectories?
    \item Compute the eigenvalues and eigenvectors of $A$.
    \item Show that $A$ is not diagonalizable.
    \item Since $A$ is not diagonalizable, the techniques we've discussed so
    far for solving differential equations like this won't work. Show that for
    any scalars $c_{1}$ and $c_{2}$, the function
    \begin{align*}
      Y(t)
      &= c_{1}e^{-t}
      \begin{bmatrix}
        1\\-1
      \end{bmatrix}
      + c_{2}e^{-t}
      \begin{bmatrix}
        1\\0
      \end{bmatrix}
      + c_{2}te^{-t}
      \begin{bmatrix}
        1\\-1
      \end{bmatrix}\\
      &=
      \begin{bmatrix}
        c_{1}e^{-t}+c_{2}e^{-t}+c_{2}te^{-t}\\
        -c_{1}e^{-t}-c_{2}te^{-t}
      \end{bmatrix}
    \end{align*}
    is a solution to the differential equation in \cref{eq:1}. (Hint:
    computing the left-hand side and right-hand side independently and then
    observe that they are equal.)
    \item \label{item:1} Using the general solution from part (e), solve the initial value
    problem
    \begin{equation*}
      \left\{ \begin{array}{l@{}l} Y'=AY \\
          Y(0)=
          \begin{bmatrix}
            1\\-1
          \end{bmatrix}
        \end{array}\right.
    \end{equation*}
  \end{enumerate}
\end{problem}

\begin{problem}[Continuation of \cref{problem:non-diagonalizable-example}]
  Let $A$ be the $2\times2$ matrix from \cref{problem:non-diagonalizable-example}.
  \begin{enumerate}[(a)]
    \item \label{item:4} Let $B=
    \begin{bmatrix}
      b_{1}\\b_{2}
    \end{bmatrix}
    \in \R^{2}.$ Using the fact that $\frac{d}{dt} \left[ e^{Mt} \right]=M e^{Mt}$
    for any square matrix, show that $Y(t) = e^{At}B$ is a solution to the
    initial value problem
    \begin{equation*}
      \left\{ \begin{array}{l@{}l} Y'=AY \\Y(0)=B  \end{array}\right.
    \end{equation*}
    \textit{(Hint: you don't need to calculate $e^{At}$ for this part. The rest of
      this problem will guide you through one way to compute $e^{At}$.)}
    \item \label{item:3} Assume that $A=PJP^{-1}$ for some $2\times 2$ matrices $J$ and
    $P$. Show that $e^{At} = Pe^{Jt}P^{-1}$.

    \textit{(Hint: use the definition $e^{M} =
      \sum_{k=0}^{\infty}\frac{M^{k}}{k!}=I+M+\frac{M^{2}}{2!}+\frac{M^{3}}{3!}+\ldots$).}
    
    \item  Let $P =
    \begin{bmatrix}
      1&1\\
      -1&0
    \end{bmatrix}
    $
    and $J =
    \begin{bmatrix}
      -1&1\\
      0&-1
    \end{bmatrix}
    $. Compute $P^{-1}$ and verify that $A = PJP^{-1}$.

    \textit{(Note: $J$ is the Jordan normal form of $A$. Since $A$ is not
      diagonalizable, $J$ is the closest we can get to diagonalizing $A$.)}
    
    \item \label{item:2} Compute $(tJ)^{k}$ for $k=0,1,2,3,4,5$. Formulate a conjecture about
    what you think $(tJ)^{k}$ is, for a general integer $k$.

    \item Using part \ref{item:2}, one can show that
    $e^{tJ} =
    \begin{bmatrix}
      e^{-t}& te^{-t}\\
      0 & e^{-t}
    \end{bmatrix} $
    (you don't have to show this). Moreover, from part \ref{item:3} we know that
    $e^{At}=Pe^{Jt}P^{-1}$. Using these two facts, compute $e^{At}$.
    \item From part \ref{item:4}, we know that $Y(t) = e^{At}B$ is a solution to the
    intial value problem. Find the solution when $B=
    \begin{bmatrix}
      1\\-1
    \end{bmatrix}
    $ and compare your answer to the answer to your answer in part
    \ref{item:1} of \cref{problem:non-diagonalizable-example}.
  \end{enumerate}
\end{problem}


\begin{problem}
Let $A =
\begin{bmatrix}
  -1&0&0\\0&0&0\\0&0&4
\end{bmatrix}
$.
\begin{enumerate}[(a)]
  \item  Determine the general solution to $Y'=AY$
  \item Solve the initial value problem $Y'=AY$ with $Y(0)=
  \begin{bmatrix}
    2\\1\\0
  \end{bmatrix} $
\end{enumerate}
\end{problem}

\begin{problem}
  Suppose a rabbit population $r$ and a wolf population $w$ are governed by
  \begin{align*}
    \left\{ \begin{array}{l@{}l} \frac{dr}{dt} = 4r-2w \\ \frac{dw}{dt} = r+w  \end{array}\right.
  \end{align*}
  \begin{enumerate}[(a)]
    \item What is the solution to this differential equation? Interpret your
    answer in terms of rabbits and wolves.
    \item If $r=300$ and $w=200$ at time $t=0$, what is the population of
    wolves and rabbits at time $t>0$?
    \item After a long time, what is the proportion of rabbits to wolves?
    Hint: compute $\lim_{t \to \infty}\frac{r(t)}{w(t)}$.
  \end{enumerate}
\end{problem}


\begin{problem}
  Find the general solution to $Y'=QY$ when $Q=
  \begin{bmatrix}
    6&0&-8\\
    -4&2&-4\\
    4&0&6
  \end{bmatrix}$ and plot the slope field of the differential equation.
\end{problem}


\begin{problem}
  Find the eigenvalues and eigenvectors, and the exponential $e^{At}$ for
  \begin{equation*}
    A =
    \begin{bmatrix}
      -1&1\\
      1&-1
    \end{bmatrix}
  \end{equation*}
  Write the general solution to $Y' = AY $, and the specific solution that
  satisfies $Y(0) =
  \begin{bmatrix}
    3\\1
  \end{bmatrix}.$ Plot the trajectory of your solution. What happens as $t \to \infty$? 
\end{problem}

\begin{problem}[Kimura 2 parameter substitution model]
  This problem introduces a model of DNA mutation that is more sophisticaed
  than Jukes Cantor, since it does not make the overly simplifying assumption
  that all nucleotides \texttt{A}, \texttt{C}, \texttt{G}, and \texttt{T}
  mutated between each other at equal rates. Instead, it assumes
  \begin{itemize}
    \item time $t$ is measured in generations
    \item mutations from \texttt{A} between \texttt{G} happen at rate $a>0$
    per generation
    \item mutations from \texttt{C} between \texttt{T} happen at rate $a>0$
    per generation
    \item mutations between all other pairs of \texttt{A}, \texttt{C},
    \texttt{G}, \texttt{T} happen at rate $b>0$ per generation
  \end{itemize}
  Let $Y(t) =
  \begin{bmatrix}
    y_{\mathtt{A}}(t)\\
    y_{\mathtt{C}}(t)\\
    y_{\mathtt{G}}(t)\\
    y_{\mathtt{T}}(t)
  \end{bmatrix},
  $ where, $y_{\mathtt{A}}(t)$ is the proportion of the DNA with
  letter \texttt{A} at time $t$, and
  $y_{\mathtt{C}},y_{\mathtt{G}}y_{\mathtt{T}}$ are defined similarly. The differential equation describing the change in $Y$ is
  \begin{equation}\label{eq:2}
    Y' = QY
  \end{equation}
  where
  \begin{equation*}
    Q =
    \begin{bmatrix}
      -a-2b& b& a& b\\
      b& -a-2b& b& a\\
      a&b&-a-2b&b\\
      b&a&b&-a-2b
    \end{bmatrix}
  \end{equation*}
  \begin{enumerate}[(a)]
    \item \label{item:5} Show the following eigenvectors are linearly independent
    \begin{equation*}
      \begin{bmatrix}
        1\\1\\1\\1
      \end{bmatrix},
      \begin{bmatrix}
        -1\\1\\-1\\1
      \end{bmatrix},
      \begin{bmatrix}
        -1\\0\\1\\0
      \end{bmatrix},
      \begin{bmatrix}
        0\\-1\\0\\1
      \end{bmatrix}
    \end{equation*}
    \item Show that the vectors from part \ref{item:5} are eigenvectors of
    $Q$. What are the corresponding eigenvalues?
    \item What is the general solution to \cref{eq:2}?
  \end{enumerate}
\end{problem}

\end{document}

% Don't write in starlight, 'Cause the words may come out real