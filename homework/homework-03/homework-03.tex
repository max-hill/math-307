\documentclass[10pt]{article}
% Math Packages
\usepackage{amsmath, mathtools}
\usepackage{amssymb}
\usepackage{amsthm}
\usepackage{amsfonts}
\usepackage{bbm}
\usepackage{breqn}
\usepackage[margin=1in]{geometry}
\usepackage{graphicx}
\usepackage{tikz}
\usetikzlibrary{arrows.meta}
\usetikzlibrary{calc}
\usepackage{forest}
\usepackage{tikz-qtree}
\graphicspath{ {./images/} }
\usepackage{hyperref}
\usepackage[capitalize]{cleveref}
\usepackage[shortlabels]{enumitem}
\usetikzlibrary{arrows,matrix,positioning}
\usepackage{multicol}


% for the pipe symbol
\usepackage[T1]{fontenc}

% Citing theorems by name. (source: https://tex.stackexchange.com/questions/109843/cleveref-and-named-theorems)
\makeatletter
\newcommand{\ncref}[1]{\cref{#1}\mynameref{#1}{\csname r@#1\endcsname}}

\def\mynameref#1#2{%
  \begingroup
  \edef\@mytxt{#2}%
  \edef\@mytst{\expandafter\@thirdoffive\@mytxt}%
  \ifx\@mytst\empty\else
  \space(\nameref{#1})\fi
  \endgroup
}
\makeatother

% Colorful Notes
\usepackage{color} \definecolor{Red}{rgb}{1,0,0} \definecolor{Blue}{rgb}{0,0,1}
\definecolor{Purple}{rgb}{.5,0,.5} \def\red{\color{Red}} \def\blue{\color{Blue}}
\def\gray{\color{gray}} \def\purple{\color{Purple}}
\newcommand{\rnote}[1]{{\red [#1]}} % \rnote{foo} gives '[foo]' in red
\newcommand{\pnote}[1]{{\purple [#1]}} % \pnote{foo} gives '[foo]' in purple
\newcommand{\bnote}[1]{{\blue #1}} % \bnote{foo} gives 'foo' in blue
\newcommand{\gnote}[1]{{\gray #1}} % \gnote{foo} gives 'foo' in gray
\newcommand{\Max}[1]{{\purple [#1]}} % \bnote{foo} then 'foo' is blue


% Claim numbering (the counter restarts after each proof environment)
\newcounter{claimcount}
\setcounter{claimcount}{0}
\newenvironment{claim}{\refstepcounter{claimcount}\par\addvspace{\medskipamount}\noindent\textbf{Claim \arabic{claimcount}:}}{}
\usepackage{etoolbox}
\AtBeginEnvironment{proof}{\setcounter{claimcount}{0}}
\newenvironment{claimproof}{\par\addvspace{\medskipamount}\noindent\textit{Proof of Claim  \arabic{claimcount}.}}{\hfill\ensuremath{\qedsymbol} \tiny{Claim}

  \medskip}
% Add claim support to cleverref
\crefname{claimcount}{Claim}{Claims}


% Math Environments
\newtheorem{theorem}{Theorem}
\newtheorem{assumption}[theorem]{Assumption}
\newtheorem{lemma}[theorem]{Lemma}
\newtheorem{proposition}[theorem]{Proposition}
\newtheorem{corollary}[theorem]{Corollary}
\newtheorem{question}[theorem]{Question}
\theoremstyle{definition}
\newtheorem{definition}[theorem]{Definition}
\newtheorem{remark}[theorem]{Remark}
\newtheorem{example}[theorem]{Example}
\newtheorem{notation}[theorem]{Notation}
\newtheorem{problem}[theorem]{Problem}

% Matrices and Column Vectors. 
\usepackage{stackengine}
\setstackgap{L}{1.0\normalbaselineskip}
\usepackage{tabstackengine}
\setstackEOL{;}% row separator
\setstackTAB{,}% column separator
\setstacktabbedgap{1ex}% inter-column gap
\setstackgap{L}{1.5\normalbaselineskip}% inter-row baselineskip
\let\nmatrix\bracketMatrixstack  %Usage: \nmatrix{1,2,3\4,5,6}
\newcommand\cv[1]{\setstackEOL{,}\bracketMatrixstack{#1}} %usage: \cv{1,2,3}

% Custom Math Coqmmands
\newcommand{\vt}{\vskip 5mm} % vertical space
\newcommand{\fl}{\noindent\textbf} % first line
\newcommand{\Fl}{\vt\noindent\textbf} % first line with space above
\newcommand{\norm}[1]{\left\lVert#1\right\rVert} % norm
\newcommand{\pnorm}[1]{\left\lVert#1\right\rVert_p} % p-norm
\newcommand{\qnorm}[1]{\left\lVert#1\right\rVert_q} % q-norm
\newcommand{\1}[1]{\textbf{1}_{\left[#1\right]}} % indicator function 
\def\limn{\lim_{n\to\infty}} % shortcut for lim as n-> infinity
\def\sumn{\sum_{n=1}^{\infty}} % shortcut for sum from n=1 to infinity
\def\sumkn{\sum_{k=1}^{n}} % shortcut for sum from k=1 to n
\def\sumin{\sum_{i=1}^{n}} % shortcut for sum from i=1 to n
\def\SAs{\sigma\text{-algebras}} % shortcut for $\sigma$-algebras
\def\SA{\sigma\text{-algebra}} % shortcut for $\sigma$-algebra
\def\Ft{\mathcal{F}_t} % time-indexed sigma-algebra (t)
\def\Fs{\mathcal{F}_s} % time-indexed sigma-algebra (s)
\def\F{\mathcal{F}} % sigma-algebra
\def\G{\mathcal{G}} % sigma-algebra
\def\R{\mathbb{R}} % Real numbers
\def\N{\mathbb{N}} % Natural numbers
\def\Z{\mathbb{Z}} % Integers
\def\E{\mathbb{E}} % Expectation
\def\P{\mathbb{P}} % Probability
\def\Q{\mathbb{Q}} % Q probability
\def\dist{\text{dist}} %Text 'dist' for things like 'dist(x,y)'
\newcommand{\indep}{\perp \!\!\! \perp}  %independence symbol
\def\Var{\mathrm{Var}} % Variance
\def\tr{\mathrm{tr}} % trace

% Brackets and Parentheses
\def\[{\left [}
    \def\]{\right ]}
% \def\({\left (}
%   \def\){\right )}



\usepackage{color}
\definecolor{Red}{rgb}{1,0,0}
\definecolor{Blue}{rgb}{0,0,1}
\definecolor{Purple}{rgb}{.5,0,.5}
\def\red{\color{Red}}
\def\blue{\color{Blue}}
\def\gray{\color{gray}}
\def\purple{\color{Purple}}
\definecolor{RoyalBlue}{cmyk}{1, 0.50, 0, 0}
\newcommand{\dempfcolor}[1]{{\color{RoyalBlue}#1}} 
\newcommand{\demph}[1]{\dempfcolor{{\sl #1}}}

% comment exactly one of the following line to show / hide the solutions
% \newcommand{\solution}[1]{{\purple #1}} % uncomment to show the solutions
\newcommand{\solution}[1]{} % uncomment to hide the solutions



\title{Lecture Notes for Math 372: \\Elementary Probability and Statistics}
\date{Last updated: \today}
% \author{mh}

\begin{document}
\begin{center}
  \section*{Math 307: Homework 03}
  \textit{Due Wednesday, September 24 (at the beginning of class)}
\end{center}


\begin{problem}[Important] %strang p80
  Let $A$ be a $3\times 3$ matrix. Suppose $x_{1},x_{2},$ and $x_{3}$ are
  column vectors such that
  \begin{equation*}
    Ax_{1} =
    \begin{bmatrix}
      1\\0\\0
    \end{bmatrix}
    \quad \text{and} \quad
    Ax_{2} =
    \begin{bmatrix}
      0\\1\\0
    \end{bmatrix}
    \quad \text{and} \quad
    Ax_{3} =
    \begin{bmatrix}
      0\\0\\1
    \end{bmatrix}.
  \end{equation*}
  If the three solutions $x_{1},x_{2}$ and $x_{3}$ are columns of a matrix
  $X$, what is $AX$?
\end{problem}



\begin{problem}
  \label{problem:thm}
  Given an $m\times n$ matrix $A=(a_{ij})$, the \demph{\textbf{transpose}} of $A$,
  denoted $A^{\top}$, is the $n\times m$ matrix $A^{\top}=(a_{ji})$. For
  example,
  \begin{equation*}
    A =
    \begin{bmatrix}
      1&2&3\\
      4&5&6
    \end{bmatrix}
    \quad \text{and} \quad
    A^{\top} =
    \begin{bmatrix}
      1&4\\
      2&5\\
      3&6
    \end{bmatrix}
  \end{equation*}
  Prove the following theorem.

  \begin{quote}\textbf{Theorem.} (Properties of the transpose).
    Suppose $A,B$ are matrices. Then whenever defined, the following
    properties hold:
    \begin{enumerate}[label=(\roman*.)]
      \item $\left( A^{\top} \right)^{\top}=A $
      \item $\left( A+B \right)^{\top} = A^{\top}+B^{\top}$
      \item $\left( cA \right)^{\top} = cA^{\top}$
      \item $\left( AB \right)^{\top}=B^{\top}A^{\top}$ \quad\textit{(this is
        sort of like the socks and shoes property)}
      \item $\left( A^{\top} \right)^{-1} = \left( A^{-1} \right)^{\top} $
    \end{enumerate}
  \end{quote}
  \textit{(This is Theorem 1.13 in the textbook. The textbook offers a proof
    of part $(iv.)$, so if you understand that proof, you can use it in your
    answer. Hint for part $(v.)$: by Theorem 22 (in Lecture 8), all you need
    to prove is $(A^{-1})^{\top}A^{\top}=I$.}
\end{problem}




\begin{problem}
  If $A=A^{\top}$, then we say that $A$ is a \demph{\textbf{symmetric matrix}}. An
  example:
  \begin{equation*}
    \begin{bmatrix}
      -5&1&2\\
      1&4&3\\
      2&3&6\\
    \end{bmatrix}
  \end{equation*}
  Note that symmetric matrices are always square.
  \begin{enumerate}[(a)]
    \item Let $A$ be any matrix. Show that $A^{\top}A$ and $AA^{\top}$ are
    both symmetric matrices. \textit{Hint: use property (iv.) from the Theorem
      in \cref{problem:thm}.}
    \item Let $A$ be a symmetric matrix. Show that if $A$ is invertible, then
    $A^{-1}$ is also symmetric. \textit{Hint: use property (v.) from the
      Theorem in \cref{problem:thm}.}
  \end{enumerate}
\end{problem}



\begin{problem}
  An \demph{\textbf{involution}} is a function $f$ such that $f(f(x))=x$ for all $x$.
  In other words, an involution is a function which is its own inverse. By
  Part (i.) in the theorem from \cref{problem:thm}, we know that matrix transposition
  is an involution, since if $f(A)=A^{\top}$, then
  \begin{equation*}
    f(f(A))=f(A^{\top})=\left( A^{\top} \right)^{\top}=A.
  \end{equation*}
  Another example is matrix inversion, since $\left( A^{-1} \right)^{-1}=A$.
  Give some other examples of involutions (from any area of math).
\end{problem}

\begin{problem}
  Let
  \begin{equation*}
    A =
    \begin{bmatrix}
      2&-1&3\\
      4&1&-2\\
      -3&2&1
    \end{bmatrix}
  \end{equation*}
  \begin{enumerate}[(a)]
    \item Find $\det(A)$ by expanding about row 1
    \item Find $\det(A)$ by expanding about row 2
    \item Find $\det(A)$ by expanding about column 1
    \item Find $\det(A)$ by expanding about column 3
  \end{enumerate}
\end{problem}

\begin{problem}
  Find the inverse of the matrix
  \begin{equation*}
    \begin{bmatrix}
      0&-2&1\\
      2&4&-1\\
      2&1&2
    \end{bmatrix}
  \end{equation*}
\end{problem}


\begin{problem}
  Solve the following linear system:
  \begin{align*}
    2x-4y+6z&=2\\
    -3x+6y-9z&=3
  \end{align*}
  Interpret your results geometrically. Provide a sketch or an image (e.g.,
  using Desmos) of the the solution.
\end{problem}

\begin{problem}
  The technical definition of ``nonsingular'' is the following:
  \begin{quote}
    \textit{Definition.} An $n\times n$ matrix $A$ is said to
    be \demph{\textbf{nonsingular}} if the only solution to the system of linear
    equations $AX=\mathbf{0}$ is $X=\mathbf{0}$.
  \end{quote}
  In other words, $A$ produces the output $\mathbf{0}$ only for the input
  $\mathbf{0}$. Note that in the above definition,
  \begin{equation*}
    X=
    \begin{bmatrix}
      x_{1}\\
      \vdots\\
      x_{n}
    \end{bmatrix}
    \quad \text{and} \quad 
    \mathbf{0}=
    \begin{bmatrix}
      0\\
      \vdots\\
      0
    \end{bmatrix}.
  \end{equation*}
  Prove that a matrix $A$ is nonsingular if and only if $\det A=0$.
  \textit{(Hint: use the key theorem of linear algebra from lecture 9).}
\end{problem}

\begin{problem}[More determinants]
  \begin{enumerate}[(a)]
    \item[]
    \item Compute the determinant by doing a cofactor expansion across an
    approriate row or column.
    \begin{equation*}
      \begin{vmatrix}
        -3&0&4\\
        2&-1&3\\
        4&0&5
      \end{vmatrix}
    \end{equation*}
    \item Compute the determinant by doing a cofactor expansion across an
    approriate row or column.
    \begin{equation*}
      \begin{vmatrix}
        6&-5&1&3\\
        3&1&-2&-1\\
        0&7&0&0\\
        3&3&0&9
      \end{vmatrix}
    \end{equation*}
    
  \end{enumerate}
  \textit{Hint: Don't try to brute force this calculation. Be clever. See
    Example 1 in section 1.5 of the textbook.}
\end{problem}


\begin{problem}[Permutation Matrices]
  A square matrix called a \demph{\textbf{permutation matrix}} if exactly one
  entry in each row and column is equal to $1$ and all other entries are $0$.
  Multiplication by such matrices permutes the rows or columns of the matrix
  multiplied. For example,
  \begin{equation*}
    \begin{bmatrix}
      0&1&0\\
      1&0&0\\
      0&0&1
    \end{bmatrix}
    \begin{bmatrix}
      1\\2\\3
    \end{bmatrix}
    =
    \begin{bmatrix}
      2\\1\\3
    \end{bmatrix}
  \end{equation*}
  Left multiplication permutes the rows (as shown above). Right multiplication
  permutes the columns.

  \begin{enumerate}[(a)]
    \item Consider the set $\left\{1,2,3,4,5\right\}$. One permutation of this
    set is $(3,2,4,1,5)$. Find the permutation matrix $P$ such that
    \begin{equation*}
      P
      \begin{bmatrix}
        1\\2\\3\\4\\5
      \end{bmatrix}
      =
      \begin{bmatrix}
        3\\2\\4\\1\\5
      \end{bmatrix}
    \end{equation*}
    \item Find a permtuation matrix $Q$ such that
    \begin{equation*}
      Q
      \begin{bmatrix}
        3\\2\\4\\1\\5
      \end{bmatrix}
      =
      \begin{bmatrix}
        1\\2\\3\\4\\5
      \end{bmatrix}
    \end{equation*}
    \item  What do you notice about the relationship between $P$ and $Q$?

    \item Give a geometric argument for why the determinant of a permutation
    matrix is always $+1$ or $-1$. (You don't need to give a proof, but try to
    be convincing.)

    % \item How many distinct $2\times 2$ permutation matrices are there? How many
    %  $3\times 3$ permutations matrices? How many $n\times n$ permutation matrices?

    % \item Argue that the product of permutation matrices is again a permutation matrix.
  \end{enumerate}
\end{problem}


\begin{problem}
  Solve the following linear system:
  \begin{align*}
    2x+3y&=5\\
    2x+y&=2\\
    x-2y&=1
  \end{align*}
  Interpret your results geometrically. Provide a sketch or an image (e.g.,
  using Desmos) of the the solution.
\end{problem}

\end{document}

% Don't write in starlight, 'Cause the words may come out real