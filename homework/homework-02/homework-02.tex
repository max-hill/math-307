\documentclass[10pt]{article}
% Math Packages
\usepackage{amsmath, mathtools}
\usepackage{amssymb}
\usepackage{amsthm}
\usepackage{amsfonts}
\usepackage{bbm}
\usepackage{breqn}
\usepackage[margin=1in]{geometry}
\usepackage{graphicx}
\usepackage{tikz}
\usetikzlibrary{arrows.meta}
\usetikzlibrary{calc}
\usepackage{forest}
\usepackage{tikz-qtree}
\graphicspath{ {./images/} }
\usepackage{hyperref}
\usepackage[capitalize]{cleveref}
\usepackage[shortlabels]{enumitem}
\usetikzlibrary{arrows,matrix,positioning}
\usepackage{multicol}


% for the pipe symbol
\usepackage[T1]{fontenc}

% Citing theorems by name. (source: https://tex.stackexchange.com/questions/109843/cleveref-and-named-theorems)
\makeatletter
\newcommand{\ncref}[1]{\cref{#1}\mynameref{#1}{\csname r@#1\endcsname}}

\def\mynameref#1#2{%
  \begingroup
  \edef\@mytxt{#2}%
  \edef\@mytst{\expandafter\@thirdoffive\@mytxt}%
  \ifx\@mytst\empty\else
  \space(\nameref{#1})\fi
  \endgroup
}
\makeatother

% Colorful Notes
\usepackage{color} \definecolor{Red}{rgb}{1,0,0} \definecolor{Blue}{rgb}{0,0,1}
\definecolor{Purple}{rgb}{.5,0,.5} \def\red{\color{Red}} \def\blue{\color{Blue}}
\def\gray{\color{gray}} \def\purple{\color{Purple}}
\newcommand{\rnote}[1]{{\red [#1]}} % \rnote{foo} gives '[foo]' in red
\newcommand{\pnote}[1]{{\purple [#1]}} % \pnote{foo} gives '[foo]' in purple
\newcommand{\bnote}[1]{{\blue #1}} % \bnote{foo} gives 'foo' in blue
\newcommand{\gnote}[1]{{\gray #1}} % \gnote{foo} gives 'foo' in gray
\newcommand{\Max}[1]{{\purple [#1]}} % \bnote{foo} then 'foo' is blue


% Claim numbering (the counter restarts after each proof environment)
\newcounter{claimcount}
\setcounter{claimcount}{0}
\newenvironment{claim}{\refstepcounter{claimcount}\par\addvspace{\medskipamount}\noindent\textbf{Claim \arabic{claimcount}:}}{}
\usepackage{etoolbox}
\AtBeginEnvironment{proof}{\setcounter{claimcount}{0}}
\newenvironment{claimproof}{\par\addvspace{\medskipamount}\noindent\textit{Proof of Claim  \arabic{claimcount}.}}{\hfill\ensuremath{\qedsymbol} \tiny{Claim}

  \medskip}
% Add claim support to cleverref
\crefname{claimcount}{Claim}{Claims}


% Math Environments
\newtheorem{theorem}{Theorem}
\newtheorem{assumption}[theorem]{Assumption}
\newtheorem{lemma}[theorem]{Lemma}
\newtheorem{proposition}[theorem]{Proposition}
\newtheorem{corollary}[theorem]{Corollary}
\newtheorem{question}[theorem]{Question}
\theoremstyle{definition}
\newtheorem{definition}[theorem]{Definition}
\newtheorem{remark}[theorem]{Remark}
\newtheorem{example}[theorem]{Example}
\newtheorem{notation}[theorem]{Notation}
\newtheorem{problem}[theorem]{Problem}

% Matrices and Column Vectors. 
\usepackage{stackengine}
\setstackgap{L}{1.0\normalbaselineskip}
\usepackage{tabstackengine}
\setstackEOL{;}% row separator
\setstackTAB{,}% column separator
\setstacktabbedgap{1ex}% inter-column gap
\setstackgap{L}{1.5\normalbaselineskip}% inter-row baselineskip
\let\nmatrix\bracketMatrixstack  %Usage: \nmatrix{1,2,3\4,5,6}
\newcommand\cv[1]{\setstackEOL{,}\bracketMatrixstack{#1}} %usage: \cv{1,2,3}

% Custom Math Coqmmands
\newcommand{\vt}{\vskip 5mm} % vertical space
\newcommand{\fl}{\noindent\textbf} % first line
\newcommand{\Fl}{\vt\noindent\textbf} % first line with space above
\newcommand{\norm}[1]{\left\lVert#1\right\rVert} % norm
\newcommand{\pnorm}[1]{\left\lVert#1\right\rVert_p} % p-norm
\newcommand{\qnorm}[1]{\left\lVert#1\right\rVert_q} % q-norm
\newcommand{\1}[1]{\textbf{1}_{\left[#1\right]}} % indicator function 
\def\limn{\lim_{n\to\infty}} % shortcut for lim as n-> infinity
\def\sumn{\sum_{n=1}^{\infty}} % shortcut for sum from n=1 to infinity
\def\sumkn{\sum_{k=1}^{n}} % shortcut for sum from k=1 to n
\def\sumin{\sum_{i=1}^{n}} % shortcut for sum from i=1 to n
\def\SAs{\sigma\text{-algebras}} % shortcut for $\sigma$-algebras
\def\SA{\sigma\text{-algebra}} % shortcut for $\sigma$-algebra
\def\Ft{\mathcal{F}_t} % time-indexed sigma-algebra (t)
\def\Fs{\mathcal{F}_s} % time-indexed sigma-algebra (s)
\def\F{\mathcal{F}} % sigma-algebra
\def\G{\mathcal{G}} % sigma-algebra
\def\R{\mathbb{R}} % Real numbers
\def\N{\mathbb{N}} % Natural numbers
\def\Z{\mathbb{Z}} % Integers
\def\E{\mathbb{E}} % Expectation
\def\P{\mathbb{P}} % Probability
\def\Q{\mathbb{Q}} % Q probability
\def\dist{\text{dist}} %Text 'dist' for things like 'dist(x,y)'
\newcommand{\indep}{\perp \!\!\! \perp}  %independence symbol
\def\Var{\mathrm{Var}} % Variance
\def\tr{\mathrm{tr}} % trace

% Brackets and Parentheses
\def\[{\left [}
    \def\]{\right ]}
% \def\({\left (}
%   \def\){\right )}



\usepackage{color}
\definecolor{Red}{rgb}{1,0,0}
\definecolor{Blue}{rgb}{0,0,1}
\definecolor{Purple}{rgb}{.5,0,.5}
\def\red{\color{Red}}
\def\blue{\color{Blue}}
\def\gray{\color{gray}}
\def\purple{\color{Purple}}
\definecolor{RoyalBlue}{cmyk}{1, 0.50, 0, 0}
\newcommand{\dempfcolor}[1]{{\color{RoyalBlue}#1}} 
\newcommand{\demph}[1]{\dempfcolor{{\sl #1}}}

% comment exactly one of the following line to show / hide the solutions
% \newcommand{\solution}[1]{{\purple #1}} % uncomment to show the solutions
\newcommand{\solution}[1]{} % uncomment to hide the solutions



\title{Lecture Notes for Math 372: \\Elementary Probability and Statistics}
\date{Last updated: \today}
% \author{mh}

\begin{document}
\begin{center}
  \section*{Math 307: Homework 02}
  \textit{Due Wednesday, September 17 (at the beginning of class)}
\end{center}
\begin{problem}
  The \demph{standard unit vectors} in $\R^{3}$ are the vectors 
  \begin{equation*}
    e_{1}  =
    \begin{bmatrix}
      1\\0\\0
    \end{bmatrix},
    \quad
    e_{2}  =
    \begin{bmatrix}
      0\\1\\0
    \end{bmatrix},
    \quad
    e_{3}  =
    \begin{bmatrix}
      0\\0\\1
    \end{bmatrix}
  \end{equation*}
  Let $A=(a_{ij})\in \mathcal{M}_{3\times 3}(\R)$. Compute the following:
  \begin{equation*}
    Ae_{1}, \quad Ae_{2}, \quad Ae_{3}.
  \end{equation*}
  What do you notice about these products?
\end{problem}


\begin{problem}
  Suppose that $A,B\in M_{n\times n}(\R).$
  \begin{enumerate}[(a)]
    \item Show that $(A+B)^{2}=A^{2}+AB+BA+B^{2}$
    \item Explain why $(A+B)^{2}$ is not equal to $A^{2}+2AB+B^{2}$ in
    general.
    \item Compute $A^{2}+AB+BA+B^{2}$ when $A=
    \begin{bmatrix}
      1&3\\
      -1&4
    \end{bmatrix}
    $
    and $B=
    \begin{bmatrix}
      0&-2\\
      1&-3
    \end{bmatrix}
    $ 
  \end{enumerate}
\end{problem}

\begin{problem}[Row reduction]
  Solve the system of linear equations using row reduction
  \begin{align*}
    3x+y-2z&=3\\
    x-8y-14z&=-14\\
    x+2y+z&=2
  \end{align*}
  Interpret your result geometrically. Provide a sketch or an image (e.g.,
  using Desmos) of the the solution.
\end{problem}


\begin{problem}[Row reduction]
  Solve the system of linear equations using row reduction
  \begin{align*}
    x+3z&=0\\
    2x+y-z&=0\\
    4x+y+5z&=0
  \end{align*}
  Interpret your result geometrically. Provide a sketch or an image (e.g.,
  using Desmos) of the the solution.
\end{problem}

\begin{problem}
  Solve the system of linear equations
  \begin{align*}
    x+2y+z=-2\\
    2x+2y-2z=3
  \end{align*}
\end{problem}

\begin{problem}
  Determine conditions on the numbers $a,b,$ and $c$ so that the following system of linear equations has
  at least one solution:
  \begin{align*}
      2x-y+3z&=a\\
      x-3y+2z&=b\\
      x+2y+z&=c
  \end{align*}
  \textit{Hint: use row reduction.}
\end{problem}

\begin{problem}
  Write the linear system in matrix form $AX=B$:
  \begin{align*}
    2x-y+4z&=1\\ 
    x+y-z&=4\\
    y+3x&=5\\
    x+y&=2
  \end{align*}
\end{problem}

\begin{problem}
  Let $f(x,y) = \frac{1}{2}(x^{2}+y^{2})$. Suppose that $f(x,y)$ represents
  the temperature of the point $(x,y)$ on the plane, in Kelvin. For example $f(1,3) =5$,
  so the temperature of the point point $(1,3)$ is 5 Kelvin.
  \begin{enumerate}[(a)]
    \item What is the coldest point on the plane? What is its temperature?
    \item Let $g(x,y)= 2x-y-5$. Use the methods of Lagrange multipliers to
    find the minimum of $f(x,y)$ subject to the constraint $g(x,y)=0$. (In
    other words, we are finding the coldest point on the line $y=2x-5$.)
    \item In part (b), you had to solve a system of equations. Identify the
    system. How many equations and how many variables does it have? Is it a
    linear?
  \end{enumerate}
\end{problem}

\begin{problem}
  Let
  \begin{equation*}
    A =
    \begin{bmatrix}
      2&1&0&0\\
      1&2&1&0\\
      0&1&2&1\\
      0&0&1&2
    \end{bmatrix}
  \end{equation*}
  \begin{enumerate}[(a)]
    \item Show that $A$ can be row-reduced to the matrix
    \begin{equation*}
      U =
      \begin{bmatrix}
        2&1&0&0\\
        0&3/2&1&0\\
        0&0&4/3&1\\
        0&0&0&5/4
      \end{bmatrix}
    \end{equation*}
    \item Find a matrix $L$ of the form
    \begin{equation*}
      L =
      \begin{bmatrix}
        x_{11}&0&0&0\\
        x_{21}&x_{22}&0&0\\
        x_{31}&x_{32}&x_{33}&0\\
        x_{41}&x_{42}&x_{43}&x_{44}\\
      \end{bmatrix}
    \end{equation*}
    such that
    \begin{equation*}
      A = LU
    \end{equation*}
    \textit{Hint: first compute the matrix product $LU$ symbolically and set
      the result equal to $A$. This will give a bunch of linear equations. Use
      the linear equations to deduce the values of the $x_{ij}$'s.}
  \end{enumerate}
\end{problem}

\begin{problem}
  A $2\times 2$ matrix $
  \begin{bmatrix}
    a&b\\
    c&d
  \end{bmatrix}
  $ is said to be \demph{positive semi-definite} if
  \begin{equation*}
    \begin{bmatrix}
      x&y
    \end{bmatrix}
    \begin{bmatrix}
      a&b\\
      c&d
    \end{bmatrix}
    \begin{bmatrix}
      x\\y
    \end{bmatrix}\geq 0
  \end{equation*}
  for all real numbers $x,y$. Show that the matrix
  \begin{equation*}
    \begin{bmatrix}
      1&1\\
      1&1
    \end{bmatrix}
  \end{equation*}
  is positive semi-definite.
\end{problem}

\begin{problem}
  Let
  \begin{equation*}
    A = \begin{bmatrix}
      1&-2&3\\
      2&-1&4\\
      1&1&1
    \end{bmatrix}
  \end{equation*}
  \begin{enumerate}[(a)]
    \item Compute the determinant of $A$.
    \item Is $A$ invertible? Justify your answer.
  \end{enumerate}
\end{problem}

\begin{problem}
  Compute the inverse of the matrix using row reduction:
  \begin{equation*}
    A =
    \begin{bmatrix}
      2&-1&3\\
      1&1&-2\\
      1&1&5
    \end{bmatrix}
  \end{equation*}
\end{problem}
\end{document}

% Don't write in starlight, 'Cause the words may come out real