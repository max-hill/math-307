\documentclass[10pt]{article}
% Math Packages
\usepackage{amsmath, mathtools}
\usepackage{amssymb}
\usepackage{amsthm}
\usepackage{amsfonts}
\usepackage{bbm}
\usepackage{breqn}
\usepackage[margin=1in]{geometry}
\usepackage{graphicx}
\usepackage{tikz}
\usetikzlibrary{arrows.meta}
\usetikzlibrary{calc}
\usepackage{forest}
\usepackage{tikz-qtree}
\graphicspath{ {./images/} }
\usepackage{hyperref}
\usepackage[capitalize]{cleveref}
\usepackage[shortlabels]{enumitem}
\usetikzlibrary{arrows,matrix,positioning}
\usepackage{multicol}


% for the pipe symbol
\usepackage[T1]{fontenc}

% Citing theorems by name. (source: https://tex.stackexchange.com/questions/109843/cleveref-and-named-theorems)
\makeatletter
\newcommand{\ncref}[1]{\cref{#1}\mynameref{#1}{\csname r@#1\endcsname}}

\def\mynameref#1#2{%
  \begingroup
  \edef\@mytxt{#2}%
  \edef\@mytst{\expandafter\@thirdoffive\@mytxt}%
  \ifx\@mytst\empty\else
  \space(\nameref{#1})\fi
  \endgroup
}
\makeatother

% Colorful Notes
\usepackage{color} \definecolor{Red}{rgb}{1,0,0} \definecolor{Blue}{rgb}{0,0,1}
\definecolor{Purple}{rgb}{.5,0,.5} \def\red{\color{Red}} \def\blue{\color{Blue}}
\def\gray{\color{gray}} \def\purple{\color{Purple}}
\newcommand{\rnote}[1]{{\red [#1]}} % \rnote{foo} gives '[foo]' in red
\newcommand{\pnote}[1]{{\purple [#1]}} % \pnote{foo} gives '[foo]' in purple
\newcommand{\bnote}[1]{{\blue #1}} % \bnote{foo} gives 'foo' in blue
\newcommand{\gnote}[1]{{\gray #1}} % \gnote{foo} gives 'foo' in gray
\newcommand{\Max}[1]{{\purple [#1]}} % \bnote{foo} then 'foo' is blue


% Claim numbering (the counter restarts after each proof environment)
\newcounter{claimcount}
\setcounter{claimcount}{0}
\newenvironment{claim}{\refstepcounter{claimcount}\par\addvspace{\medskipamount}\noindent\textbf{Claim \arabic{claimcount}:}}{}
\usepackage{etoolbox}
\AtBeginEnvironment{proof}{\setcounter{claimcount}{0}}
\newenvironment{claimproof}{\par\addvspace{\medskipamount}\noindent\textit{Proof of Claim  \arabic{claimcount}.}}{\hfill\ensuremath{\qedsymbol} \tiny{Claim}

  \medskip}
% Add claim support to cleverref
\crefname{claimcount}{Claim}{Claims}


% Math Environments
\newtheorem{theorem}{Theorem}
\newtheorem{assumption}[theorem]{Assumption}
\newtheorem{lemma}[theorem]{Lemma}
\newtheorem{proposition}[theorem]{Proposition}
\newtheorem{corollary}[theorem]{Corollary}
\newtheorem{question}[theorem]{Question}
\theoremstyle{definition}
\newtheorem{definition}[theorem]{Definition}
\newtheorem{remark}[theorem]{Remark}
\newtheorem{example}[theorem]{Example}
\newtheorem{notation}[theorem]{Notation}
\newtheorem{problem}[theorem]{Problem}

% Matrices and Column Vectors. 
\usepackage{stackengine}
\setstackgap{L}{1.0\normalbaselineskip}
\usepackage{tabstackengine}
\setstackEOL{;}% row separator
\setstackTAB{,}% column separator
\setstacktabbedgap{1ex}% inter-column gap
\setstackgap{L}{1.5\normalbaselineskip}% inter-row baselineskip
\let\nmatrix\bracketMatrixstack  %Usage: \nmatrix{1,2,3\4,5,6}
\newcommand\cv[1]{\setstackEOL{,}\bracketMatrixstack{#1}} %usage: \cv{1,2,3}

% Custom Math Coqmmands
\newcommand{\vt}{\vskip 5mm} % vertical space
\newcommand{\fl}{\noindent\textbf} % first line
\newcommand{\Fl}{\vt\noindent\textbf} % first line with space above
\newcommand{\norm}[1]{\left\lVert#1\right\rVert} % norm
\newcommand{\pnorm}[1]{\left\lVert#1\right\rVert_p} % p-norm
\newcommand{\qnorm}[1]{\left\lVert#1\right\rVert_q} % q-norm
\newcommand{\1}[1]{\textbf{1}_{\left[#1\right]}} % indicator function 
\def\limn{\lim_{n\to\infty}} % shortcut for lim as n-> infinity
\def\sumn{\sum_{n=1}^{\infty}} % shortcut for sum from n=1 to infinity
\def\sumkn{\sum_{k=1}^{n}} % shortcut for sum from k=1 to n
\def\sumin{\sum_{i=1}^{n}} % shortcut for sum from i=1 to n
\def\SAs{\sigma\text{-algebras}} % shortcut for $\sigma$-algebras
\def\SA{\sigma\text{-algebra}} % shortcut for $\sigma$-algebra
\def\Ft{\mathcal{F}_t} % time-indexed sigma-algebra (t)
\def\Fs{\mathcal{F}_s} % time-indexed sigma-algebra (s)
\def\F{\mathcal{F}} % sigma-algebra
\def\G{\mathcal{G}} % sigma-algebra
\def\R{\mathbb{R}} % Real numbers
\def\N{\mathbb{N}} % Natural numbers
\def\Z{\mathbb{Z}} % Integers
\def\E{\mathbb{E}} % Expectation
\def\P{\mathbb{P}} % Probability
\def\Q{\mathbb{Q}} % Q probability
\def\dist{\text{dist}} %Text 'dist' for things like 'dist(x,y)'
\newcommand{\indep}{\perp \!\!\! \perp}  %independence symbol
\def\Var{\mathrm{Var}} % Variance
\def\tr{\mathrm{tr}} % trace

% Brackets and Parentheses
\def\[{\left [}
    \def\]{\right ]}
% \def\({\left (}
%   \def\){\right )}



\usepackage{color}
\definecolor{Red}{rgb}{1,0,0}
\definecolor{Blue}{rgb}{0,0,1}
\definecolor{Purple}{rgb}{.5,0,.5}
\def\red{\color{Red}}
\def\blue{\color{Blue}}
\def\gray{\color{gray}}
\def\purple{\color{Purple}}
\definecolor{RoyalBlue}{cmyk}{1, 0.50, 0, 0}
\newcommand{\dempfcolor}[1]{{\color{RoyalBlue}#1}} 
\newcommand{\demph}[1]{\dempfcolor{{\sl #1}}}

% comment exactly one of the following line to show / hide the solutions
% \newcommand{\solution}[1]{{\purple #1}} % uncomment to show the solutions
\newcommand{\solution}[1]{} % uncomment to hide the solutions



\title{Lecture Notes for Math 372: \\Elementary Probability and Statistics}
\date{Last updated: \today}
% \author{mh}

\begin{document}
\begin{center}
  \section*{Math 307: Homework 06}
  \textit{Due Wednesday, October 15 (at the beginning of class). This homework
  is mostly based on sections 2.3 and 2.4 in the textbook.}
\end{center}




\begin{problem}
  Is $
  \begin{bmatrix}
    3&5\\3&4
  \end{bmatrix}
  $ in $\text{Span} \left\{
    \begin{bmatrix}
      1&1\\
      0&1
    \end{bmatrix},
    \begin{bmatrix}
      1&1\\
      1&0
    \end{bmatrix},
    \begin{bmatrix}
      1&0\\
      0&-1
    \end{bmatrix}
  \right\}$?
\end{problem}


\begin{problem}
  Find the determinant of the matrix $
  \begin{bmatrix}
    4&3&2&1\\
    -2&5&-1&-2\\
    0&1&0&0\\
    0&2&0&-2
  \end{bmatrix}.
  $
\end{problem}

\begin{problem}
  \begin{enumerate}[(a)]
    \item []
    \item Find the inverse of the matrix $
    \begin{bmatrix}
      0&-2&1\\
      2&4&-1\\
      2&1&2
    \end{bmatrix}
    $ using row reduction.
    \item Use the inverse matrix from part (a) to solve the linear system
    \begin{equation*}
      \left\{ \begin{array}{l@{}l}
          -2y+z=2 \\
          2x+4y-z=-1\\
          2x+y+2z=5
        \end{array}\right.
    \end{equation*}
  \end{enumerate}
\end{problem}
\begin{problem}
  Show that $
  \begin{bmatrix}
    2\\-1\\0
  \end{bmatrix},
  \begin{bmatrix}
    1\\3\\-1
  \end{bmatrix},
  \begin{bmatrix}
    1\\-4\\-1
  \end{bmatrix}
  $
  form a basis for $\R^{3}$.
\end{problem}



\begin{problem}
  For each part, determine whether the vectors form a basis for $\R^{3}$.
  \begin{enumerate}[(a)]
    \item $
    \begin{bmatrix}
      1\\1\\4
    \end{bmatrix},
    \begin{bmatrix}
      2\\-1\\0
    \end{bmatrix},
    \begin{bmatrix}
      0\\-1\\8
    \end{bmatrix}
    $
    \item $
    \begin{bmatrix}
      1\\1\\1
    \end{bmatrix},
    \begin{bmatrix}
      -3\\2\\5
    \end{bmatrix},
    \begin{bmatrix}
      4\\4\\5
    \end{bmatrix},
    \begin{bmatrix}
      0\\-3\\1
    \end{bmatrix}
    $
    \item $
    \begin{bmatrix}
      -7\\-9\\1
    \end{bmatrix},
    \begin{bmatrix}
      1\\2\\2
    \end{bmatrix}
    $
  \end{enumerate}
\end{problem}
\noindent \textbf{Instructions for for problems 3-6:}
\begin{enumerate}[(a)]
  \item Find a basis for the nullspace of the matrix
  \item Find a basis for the row space of the matrix
  \item Find a basis for the column space of the matrix
  \item Determine the rank of the matrix
  \item Determine whether the matrix is invertible
\end{enumerate}
\textit{Note that parts (a)-(c) do not have unique answers.}

\begin{problem}
  $
    \begin{bmatrix}
      1&1\\1&-2
    \end{bmatrix}
  $
\end{problem}

\begin{problem}
  $
    \begin{bmatrix}
      1&-1&1\\-1&1&0\\1&-1&2
    \end{bmatrix}
  $
\end{problem}

\begin{problem}
  $
    \begin{bmatrix}
      2&-1&0\\1&1&-1\\1&0&1
    \end{bmatrix}
  $
\end{problem}

\begin{problem}
  $
    \begin{bmatrix}
      1&1&0&3\\
      1&1&1&-2\\
      3&3&2&-1
    \end{bmatrix}
  $
\end{problem}



\begin{problem}
  Determine whether the given vectors are linearly dependent or linearly
  independent.
  \begin{enumerate}[(a)]
    \item $\begin{bmatrix}
      1&0\\
      -1&1
    \end{bmatrix},
    \begin{bmatrix}
      0&1\\
      1&0
    \end{bmatrix},
    \begin{bmatrix}
      1&1\\
      1&1
    \end{bmatrix}$
    \item $x^{3}-1$, $x^{2}-1$, $x-1$, $1$. 
  \end{enumerate}
\end{problem}
% \begin{problem}[The matrix exponential]
%   Recall that for any real number $x$, the \demph{exponential function}
%   $e^{x}$ is defined by the power series
%   \begin{equation}\label{eq:1}
%     e^{x} = \sum_{k=0}^{\infty}\frac{x^{k}}{k!}.
%   \end{equation}
%   \begin{enumerate}[(a)]
%     \item Remarkably, the exponential function is defined even if $x$ is a
%     matrix. In particular, letting $A$ be any square matrix, we can define the
%     \demph{matrix exponential} as
%     \begin{equation*}
%       e^{A} = \sum_{k=0}^{\infty} \frac{A^{k}}{k!}.
%     \end{equation*}
%     Note that $e^{A}$ is a square matrix of the same dimensions as $A$.
%     Compute $e^{A}$ when $A =
%     \begin{bmatrix}
%       2&0\\
%       0&3
%     \end{bmatrix}
%     $. 
%     \item Let $b$ and $t$ be real numbers. By differentiating the series in
%     \cref{eq:1} term-by-term (with $x=bt$), show that
%     \begin{equation*}
%       \frac{d}{dt} \left[ e^{bt}  \right] = b e^{bt}.
%     \end{equation*}
%     \item Let $B$ be any square matrix, and let $t$ be a real number.
%     Following your approach in part (b), show that
%     \begin{equation*}
%       \frac{d}{dt} \left[ e^{Bt} \right]  = B e^{Bt}.
%     \end{equation*}
%   \end{enumerate}
% \end{problem}



\end{document}

% Don't write in starlight, 'Cause the words may come out real