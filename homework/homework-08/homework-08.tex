\documentclass[10pt]{article}
% Math Packages
\usepackage{amsmath, mathtools}
\usepackage{amssymb}
\usepackage{amsthm}
\usepackage{amsfonts}
\usepackage{bbm}
\usepackage{breqn}
\usepackage[margin=1in]{geometry}
\usepackage{graphicx}
\usepackage{tikz}
\usetikzlibrary{arrows.meta}
\usetikzlibrary{calc}
\usepackage{forest}
\usepackage{tikz-qtree}
\graphicspath{ {./images/} }
\usepackage{hyperref}
\usepackage[capitalize]{cleveref}
\usepackage[shortlabels]{enumitem}
\usetikzlibrary{arrows,matrix,positioning}
\usepackage{multicol}


% for the pipe symbol
\usepackage[T1]{fontenc}

% Citing theorems by name. (source: https://tex.stackexchange.com/questions/109843/cleveref-and-named-theorems)
\makeatletter
\newcommand{\ncref}[1]{\cref{#1}\mynameref{#1}{\csname r@#1\endcsname}}

\def\mynameref#1#2{%
  \begingroup
  \edef\@mytxt{#2}%
  \edef\@mytst{\expandafter\@thirdoffive\@mytxt}%
  \ifx\@mytst\empty\else
  \space(\nameref{#1})\fi
  \endgroup
}
\makeatother

% Colorful Notes
\usepackage{color} \definecolor{Red}{rgb}{1,0,0} \definecolor{Blue}{rgb}{0,0,1}
\definecolor{Purple}{rgb}{.5,0,.5} \def\red{\color{Red}} \def\blue{\color{Blue}}
\def\gray{\color{gray}} \def\purple{\color{Purple}}
\newcommand{\rnote}[1]{{\red [#1]}} % \rnote{foo} gives '[foo]' in red
\newcommand{\pnote}[1]{{\purple [#1]}} % \pnote{foo} gives '[foo]' in purple
\newcommand{\bnote}[1]{{\blue #1}} % \bnote{foo} gives 'foo' in blue
\newcommand{\gnote}[1]{{\gray #1}} % \gnote{foo} gives 'foo' in gray
\newcommand{\Max}[1]{{\purple [#1]}} % \bnote{foo} then 'foo' is blue


% Claim numbering (the counter restarts after each proof environment)
\newcounter{claimcount}
\setcounter{claimcount}{0}
\newenvironment{claim}{\refstepcounter{claimcount}\par\addvspace{\medskipamount}\noindent\textbf{Claim \arabic{claimcount}:}}{}
\usepackage{etoolbox}
\AtBeginEnvironment{proof}{\setcounter{claimcount}{0}}
\newenvironment{claimproof}{\par\addvspace{\medskipamount}\noindent\textit{Proof of Claim  \arabic{claimcount}.}}{\hfill\ensuremath{\qedsymbol} \tiny{Claim}

  \medskip}
% Add claim support to cleverref
\crefname{claimcount}{Claim}{Claims}


% Math Environments
\newtheorem{theorem}{Theorem}
\newtheorem{assumption}[theorem]{Assumption}
\newtheorem{lemma}[theorem]{Lemma}
\newtheorem{proposition}[theorem]{Proposition}
\newtheorem{corollary}[theorem]{Corollary}
\newtheorem{question}[theorem]{Question}
\theoremstyle{definition}
\newtheorem{definition}[theorem]{Definition}
\newtheorem{remark}[theorem]{Remark}
\newtheorem{example}[theorem]{Example}
\newtheorem{notation}[theorem]{Notation}
\newtheorem{problem}[theorem]{Problem}

% Matrices and Column Vectors. 
\usepackage{stackengine}
\setstackgap{L}{1.0\normalbaselineskip}
\usepackage{tabstackengine}
\setstackEOL{;}% row separator
\setstackTAB{,}% column separator
\setstacktabbedgap{1ex}% inter-column gap
\setstackgap{L}{1.5\normalbaselineskip}% inter-row baselineskip
\let\nmatrix\bracketMatrixstack  %Usage: \nmatrix{1,2,3\4,5,6}
\newcommand\cv[1]{\setstackEOL{,}\bracketMatrixstack{#1}} %usage: \cv{1,2,3}

% Custom Math Coqmmands
\newcommand{\vt}{\vskip 5mm} % vertical space
\newcommand{\fl}{\noindent\textbf} % first line
\newcommand{\Fl}{\vt\noindent\textbf} % first line with space above
\newcommand{\norm}[1]{\left\lVert#1\right\rVert} % norm
\newcommand{\pnorm}[1]{\left\lVert#1\right\rVert_p} % p-norm
\newcommand{\qnorm}[1]{\left\lVert#1\right\rVert_q} % q-norm
\newcommand{\1}[1]{\textbf{1}_{\left[#1\right]}} % indicator function 
\def\limn{\lim_{n\to\infty}} % shortcut for lim as n-> infinity
\def\sumn{\sum_{n=1}^{\infty}} % shortcut for sum from n=1 to infinity
\def\sumkn{\sum_{k=1}^{n}} % shortcut for sum from k=1 to n
\def\sumin{\sum_{i=1}^{n}} % shortcut for sum from i=1 to n
\def\SAs{\sigma\text{-algebras}} % shortcut for $\sigma$-algebras
\def\SA{\sigma\text{-algebra}} % shortcut for $\sigma$-algebra
\def\Ft{\mathcal{F}_t} % time-indexed sigma-algebra (t)
\def\Fs{\mathcal{F}_s} % time-indexed sigma-algebra (s)
\def\F{\mathcal{F}} % sigma-algebra
\def\G{\mathcal{G}} % sigma-algebra
\def\R{\mathbb{R}} % Real numbers
\def\N{\mathbb{N}} % Natural numbers
\def\Z{\mathbb{Z}} % Integers
\def\E{\mathbb{E}} % Expectation
\def\P{\mathbb{P}} % Probability
\def\Q{\mathbb{Q}} % Q probability
\def\dist{\text{dist}} %Text 'dist' for things like 'dist(x,y)'
\newcommand{\indep}{\perp \!\!\! \perp}  %independence symbol
\def\Var{\mathrm{Var}} % Variance
\def\tr{\mathrm{tr}} % trace

% Brackets and Parentheses
\def\[{\left [}
    \def\]{\right ]}
% \def\({\left (}
%   \def\){\right )}



\usepackage{color}
\definecolor{Red}{rgb}{1,0,0}
\definecolor{Blue}{rgb}{0,0,1}
\definecolor{Purple}{rgb}{.5,0,.5}
\def\red{\color{Red}}
\def\blue{\color{Blue}}
\def\gray{\color{gray}}
\def\purple{\color{Purple}}
\definecolor{RoyalBlue}{cmyk}{1, 0.50, 0, 0}
\newcommand{\dempfcolor}[1]{{\color{RoyalBlue}#1}} 
\newcommand{\demph}[1]{\dempfcolor{{\sl #1}}}

% comment exactly one of the following line to show / hide the solutions
% \newcommand{\solution}[1]{{\purple #1}} % uncomment to show the solutions
\newcommand{\solution}[1]{} % uncomment to hide the solutions



\title{Lecture Notes for Math 372: \\Elementary Probability and Statistics}
\date{Last updated: \today}
% \author{mh}

\begin{document}
\begin{center}
  \section*{Math 307: Homework 08}
  \textit{Due Wednesday, October 29. This homework is mostly based on section
    5.1 in the textbook.}
\end{center}

\begin{problem}
  Find all solutions of
  \begin{equation*}
    \left\{ \begin{array}{l@{}l}
        x_{1}+x_{2}+x_{3}+x_{4}=0\\
        x_{1}+x_{2}+x_{3}-x_{4}=4\\
        x_{1}+x_{2}-x_{3}+x_{4}=-4\\
        x_{1}-x_{2}+x_{3}+x_{4}=2\\
      \end{array}\right.
  \end{equation*}
\end{problem}


\begin{problem}
  Let
  \begin{equation*}
    A=
    \begin{bmatrix}
      4&-1&2&1\\
      2&3&-1&-2\\
      0&7&-4&-5\\
      2&-11&7&8
    \end{bmatrix}
  \end{equation*}
  For the following parts, you must fully justify your work to recieve
  credit---this includes providing at least one or two complete sentences
  explaining why your calculations justify your answer.
  \begin{enumerate}[(a)]
    \item Find a basis for $NS(A)$
    \item Find a basis for $RS(A)$
    \item Find a basis for $CS(A)$
    \item Determine the rank of $A$.
    \item Determine whether the matrix is invertible
  \end{enumerate}
\end{problem}

\begin{problem}
  The matrix $A=
  \begin{bmatrix}
    2&0\\
    0&1
  \end{bmatrix}
  $ produces a \demph{stretching} in the $x$-direction. Draw the circle
  $x^{2}+y^{2}=1$. What happens to this circle when space is transformed by
  $A$? Illustrate your answer with a sketch.
\end{problem}

\begin{problem}
  The matrix $A=
  \begin{bmatrix}
    1&0\\
    3&1
  \end{bmatrix}
  $ yields a \demph{shearing} transformation, which leaves the $y$-axis
  unchanged. Sketch its effect on the $x$-axis, by indicating what happens to
  the points $(x,y)= (1,0),(2,0)$, and $(-1,0)$---and how the whole axis is
  transformed.
\end{problem}

\begin{problem}
  \begin{enumerate}[(a)]
    \item[]
    \item What $2\times 2$ matrix has the effect of rotating every vector
    counterclockwise $90^{\circ}$ and then projecting the result onto the
    $x$-axis?
    \item What $2\times 2$ matrix represents projection onto the $x$-axis
    followed by projection onto the $y$-axis?
  \end{enumerate}
\end{problem}

\begin{problem}
  Let $c$ be a scalar, and let $T:V \to W$ be a linear transformation. Verify
  that $cT$ is a linear transformation.
\end{problem}


\begin{problem}
  Let $S,T:\R^{2} \to \R^{2}$ be the linear transformations
  \begin{equation*}
    S
    \begin{bmatrix}
      x\\y
    \end{bmatrix}
    =
    \begin{bmatrix}
      2x-y\\x+2y
    \end{bmatrix}
    \quad \text{and} \quad
    T
    \begin{bmatrix}
      x\\y
    \end{bmatrix}
    =
    \begin{bmatrix}
      x+3y\\x-y
    \end{bmatrix}.
  \end{equation*}
  \begin{enumerate}[(a)]
    \item Find matrices $A,B$ such that $T$ and $S$ are expressed as the
    matrix transformations $T(X)=AX$ and $S(X) = BX$.
    \item Find the matrix $C$ such that the composition $S\circ T$ is
    expressed in the form $(S\circ T)(X)=CX$. Then verify that $C=AB$.
    \item Find the matrix $D$ such that the composition $T\circ S$ is
    expressed in the form $(T\circ S)(X)=DX$. Then verify that $D=BA$.
  \end{enumerate}
\end{problem}

\begin{problem}
  The $4$ Hadamard matrix is
  \begin{equation*}
    H=
    \begin{bmatrix}
      1&1&1&1\\
      1&-1&1&-1\\
      1&1&-1&-1\\
      1&-1&-1&1
    \end{bmatrix}
  \end{equation*}
  Find $H^{-1}$ and write $v=
  \begin{bmatrix}
    7\\5\\3\\1
  \end{bmatrix}
  $
  as a linear combination of the columns of $H$.
\end{problem}

\begin{problem}
  Find the rank and nullsapce of
  \begin{equation*}
    A=
    \begin{bmatrix}
      0&0&1\\
      0&0&1\\
      1&1&1
    \end{bmatrix}
    \quad \text{and} \quad
    B =
    \begin{bmatrix}
      0&0&1&2\\
      0&0&1&2\\
      1&1&1&0
    \end{bmatrix}
  \end{equation*}
\end{problem}

\begin{problem}
  If $S$ and $T$ are linear transformations with $S(v)=T(v)=v$, then
  $S(T(v))=v$ or $v^{2}$?
\end{problem}

\begin{problem}
  A linear transformation must leave the zero vector fixed: $T(0)=0$. Prove
  this from $T(u+v)=T(u)+T(v)$ by choosing $v=\underline{\phantom{hello}}$.
  Prove it also from the requirement $T(cv)=cT(v)$ by choosing
  $c=\underline{\phantom{hello}}$.
\end{problem}


\begin{problem}
  Every straight line remains straight after a linear transformation. If $z$
  is halfway between $x$ and $y$, show that $Az$ is halfway between $Ax$ and
  $Ay$
\end{problem}

\begin{problem}
  True or false, with counterexample if false:
  \begin{enumerate}[(a)]
    \item If the vectors $x_{1},\ldots,x_{m}$ span a subspace $S$, then $\dim
    S =m.$
    \item The intersection of two subspaces of a vector space cannot be
    emptpy.
    \item If $Ax=Ay$, then $x=y$.
    \item The row space of $A$ has a unique basis that can be computed by
    reducing $A$ to reduced row-echelon form.
    \item If a square matrix $A$ has independent columns, then so does $A^{2}$.
  \end{enumerate}
\end{problem}

\end{document}

% Don't write in starlight, 'Cause the words may come out real