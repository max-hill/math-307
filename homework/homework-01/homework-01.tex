\documentclass[10pt]{article}
% Math Packages
\usepackage{amsmath, mathtools}
\usepackage{amssymb}
\usepackage{amsthm}
\usepackage{amsfonts}
\usepackage{bbm}
\usepackage{breqn}
\usepackage[margin=1in]{geometry}
\usepackage{graphicx}
\usepackage{tikz}
\usetikzlibrary{arrows.meta}
\usetikzlibrary{calc}
\usepackage{forest}
\usepackage{tikz-qtree}
\graphicspath{ {./images/} }
\usepackage{hyperref}
\usepackage[capitalize]{cleveref}
\usepackage[shortlabels]{enumitem}
\usetikzlibrary{arrows,matrix,positioning}
\usepackage{multicol}


% for the pipe symbol
\usepackage[T1]{fontenc}

% Citing theorems by name. (source: https://tex.stackexchange.com/questions/109843/cleveref-and-named-theorems)
\makeatletter
\newcommand{\ncref}[1]{\cref{#1}\mynameref{#1}{\csname r@#1\endcsname}}

\def\mynameref#1#2{%
  \begingroup
  \edef\@mytxt{#2}%
  \edef\@mytst{\expandafter\@thirdoffive\@mytxt}%
  \ifx\@mytst\empty\else
  \space(\nameref{#1})\fi
  \endgroup
}
\makeatother

% Colorful Notes
\usepackage{color} \definecolor{Red}{rgb}{1,0,0} \definecolor{Blue}{rgb}{0,0,1}
\definecolor{Purple}{rgb}{.5,0,.5} \def\red{\color{Red}} \def\blue{\color{Blue}}
\def\gray{\color{gray}} \def\purple{\color{Purple}}
\newcommand{\rnote}[1]{{\red [#1]}} % \rnote{foo} gives '[foo]' in red
\newcommand{\pnote}[1]{{\purple [#1]}} % \pnote{foo} gives '[foo]' in purple
\newcommand{\bnote}[1]{{\blue #1}} % \bnote{foo} gives 'foo' in blue
\newcommand{\gnote}[1]{{\gray #1}} % \gnote{foo} gives 'foo' in gray
\newcommand{\Max}[1]{{\purple [#1]}} % \bnote{foo} then 'foo' is blue


% Claim numbering (the counter restarts after each proof environment)
\newcounter{claimcount}
\setcounter{claimcount}{0}
\newenvironment{claim}{\refstepcounter{claimcount}\par\addvspace{\medskipamount}\noindent\textbf{Claim \arabic{claimcount}:}}{}
\usepackage{etoolbox}
\AtBeginEnvironment{proof}{\setcounter{claimcount}{0}}
\newenvironment{claimproof}{\par\addvspace{\medskipamount}\noindent\textit{Proof of Claim  \arabic{claimcount}.}}{\hfill\ensuremath{\qedsymbol} \tiny{Claim}

  \medskip}
% Add claim support to cleverref
\crefname{claimcount}{Claim}{Claims}


% Math Environments
\newtheorem{theorem}{Theorem}
\newtheorem{assumption}[theorem]{Assumption}
\newtheorem{lemma}[theorem]{Lemma}
\newtheorem{proposition}[theorem]{Proposition}
\newtheorem{corollary}[theorem]{Corollary}
\newtheorem{question}[theorem]{Question}
\theoremstyle{definition}
\newtheorem{definition}[theorem]{Definition}
\newtheorem{remark}[theorem]{Remark}
\newtheorem{example}[theorem]{Example}
\newtheorem{notation}[theorem]{Notation}
\newtheorem{problem}[theorem]{Problem}

% Matrices and Column Vectors. 
\usepackage{stackengine}
\setstackgap{L}{1.0\normalbaselineskip}
\usepackage{tabstackengine}
\setstackEOL{;}% row separator
\setstackTAB{,}% column separator
\setstacktabbedgap{1ex}% inter-column gap
\setstackgap{L}{1.5\normalbaselineskip}% inter-row baselineskip
\let\nmatrix\bracketMatrixstack  %Usage: \nmatrix{1,2,3\4,5,6}
\newcommand\cv[1]{\setstackEOL{,}\bracketMatrixstack{#1}} %usage: \cv{1,2,3}

% Custom Math Coqmmands
\newcommand{\vt}{\vskip 5mm} % vertical space
\newcommand{\fl}{\noindent\textbf} % first line
\newcommand{\Fl}{\vt\noindent\textbf} % first line with space above
\newcommand{\norm}[1]{\left\lVert#1\right\rVert} % norm
\newcommand{\pnorm}[1]{\left\lVert#1\right\rVert_p} % p-norm
\newcommand{\qnorm}[1]{\left\lVert#1\right\rVert_q} % q-norm
\newcommand{\1}[1]{\textbf{1}_{\left[#1\right]}} % indicator function 
\def\limn{\lim_{n\to\infty}} % shortcut for lim as n-> infinity
\def\sumn{\sum_{n=1}^{\infty}} % shortcut for sum from n=1 to infinity
\def\sumkn{\sum_{k=1}^{n}} % shortcut for sum from k=1 to n
\def\sumin{\sum_{i=1}^{n}} % shortcut for sum from i=1 to n
\def\SAs{\sigma\text{-algebras}} % shortcut for $\sigma$-algebras
\def\SA{\sigma\text{-algebra}} % shortcut for $\sigma$-algebra
\def\Ft{\mathcal{F}_t} % time-indexed sigma-algebra (t)
\def\Fs{\mathcal{F}_s} % time-indexed sigma-algebra (s)
\def\F{\mathcal{F}} % sigma-algebra
\def\G{\mathcal{G}} % sigma-algebra
\def\R{\mathbb{R}} % Real numbers
\def\N{\mathbb{N}} % Natural numbers
\def\Z{\mathbb{Z}} % Integers
\def\E{\mathbb{E}} % Expectation
\def\P{\mathbb{P}} % Probability
\def\Q{\mathbb{Q}} % Q probability
\def\dist{\text{dist}} %Text 'dist' for things like 'dist(x,y)'
\newcommand{\indep}{\perp \!\!\! \perp}  %independence symbol
\def\Var{\mathrm{Var}} % Variance
\def\tr{\mathrm{tr}} % trace

% Brackets and Parentheses
\def\[{\left [}
    \def\]{\right ]}
% \def\({\left (}
%   \def\){\right )}



\usepackage{color}
\definecolor{Red}{rgb}{1,0,0}
\definecolor{Blue}{rgb}{0,0,1}
\definecolor{Purple}{rgb}{.5,0,.5}
\def\red{\color{Red}}
\def\blue{\color{Blue}}
\def\gray{\color{gray}}
\def\purple{\color{Purple}}
\definecolor{RoyalBlue}{cmyk}{1, 0.50, 0, 0}
\newcommand{\dempfcolor}[1]{{\color{RoyalBlue}#1}} 
\newcommand{\demph}[1]{\dempfcolor{{\sl #1}}}

% comment exactly one of the following line to show / hide the solutions
% \newcommand{\solution}[1]{{\purple #1}} % uncomment to show the solutions
\newcommand{\solution}[1]{} % uncomment to hide the solutions



\title{Lecture Notes for Math 372: \\Elementary Probability and Statistics}
\date{Last updated: \today}
% \author{mh}

\begin{document}
\begin{center}
  \section*{Math 307: Homework 01}
  \textit{Due Wednesday, September 10 (at the beginning of class)}
\end{center}


\begin{problem}%[lagrange - UCR]
  Consider the function $f(x,y,z) = x^{2}+yz$, which is defined for all
  $x,y,z\in \R$.
  \begin{enumerate}[(a)]
    \item Use the method of Lagrange multipliers to find
    the critical points of $f$ subject to the constraint
    \begin{equation*}
      xy+xz-2=0.
    \end{equation*}
    You don't have to classify the critical points, just find them. If you
    don't recall the method of Lagrange multipliers, a good resources is
    \url{https://math.libretexts.org/Bookshelves/Calculus/Calculus_(OpenStax)/14%3A_Differentiation_of_Functions_of_Several_Variables/14.08%3A_Lagrange_Multipliers}.
    
    \item Your solution to part (a) involved solving a system of linear
    equations. Identify the system. How many equations does it have? How many
    variables?
  \end{enumerate}
\end{problem}

\begin{problem}
  In Section 2.1 of the lecture notes (lecture 2), we introduced a trichotomy
  for solution sets of a system of linear equations. For each of \textbf{Cases
    1}, \textbf{Case 2}, and \textbf{Case 3} from that section, give an
  example of a linear system of equations with two equations and two variables
  $x$ and $y$. Plot and label the lines.
\end{problem}

\begin{problem}
  Consider a $2\times 2$  matrix $\begin{bmatrix}
    a_{11}&a_{12}\\
    a_{21}&a_{22}\\
  \end{bmatrix}$, where $a_{11},a_{12},a_{21},a_{22}\in \R$ and assume that $a_{12}$ and
  $a_{22}$ are both nonzero. Show that the following are equivalent:
  \begin{enumerate}
    \item[(i)] The determinant\footnote{The \demph{determinant} of the
      $2\times 2$ matrix $
      \begin{bmatrix}
        a_{11}&a_{12}\\
        a_{21}&a_{22}\\
      \end{bmatrix}$
      is the quantity $a_{11}a_{22}-a_{12}a_{21}$.} of the matrix is nonzero.
    \item[(ii)] The linear system
    $\left\{ \begin{array}{l@{}l} a_{11}x+a_{12}y=b_{1} \\
        a_{21}x+a_{22}y=b_{2} \end{array}\right.$ has exactly one solution.
  \end{enumerate}
\end{problem}


\begin{problem}[Row reduction]
  Solve the system of equations using row reduction
  \begin{align*}
    x+y-z&=0\\
    2x+3y-2z&=6\\
    x+2y+2z&=10
  \end{align*}
  Interpret your result geometrically. Provide a sketch or an image (e.g.,
  using Desmos) of the the solution
\end{problem}

\begin{problem}[Row reduction]
  Solve the system of equations using row reduction
  \begin{align*}
    2x+y-2z&=0\\
    2x-y-2z&=0\\
    x+2y-4z&=0
  \end{align*}
  Interpret your result geometrically. Provide a sketch or an image (e.g.,
  using Desmos) of the the solution
\end{problem}


\begin{problem}[Row reduction]
  Solve the system of equations using row reduction
  \begin{align*}
    2x+3y-4z&=3\\
    2x+3y-2z&=3\\
    4x+6y-2z&=7
  \end{align*}
  Interpret your result geometrically. Provide a sketch or an image (e.g.,
  using Desmos) of the the solution
\end{problem}

\begin{problem}
  Define the following matrices
  \begin{equation*}
    A =
    \begin{bmatrix}
      1&2\\
      3&-1\\
      2&-1
    \end{bmatrix}
    \quad
    B =
    \begin{bmatrix}
      2&-1\\
      -3&-2\\
      0&4
    \end{bmatrix}
    \quad
    C =
    \begin{bmatrix}
      2&-1\\
      1&5
    \end{bmatrix}
    \quad
    D =
    \begin{bmatrix}
      0&1\\
      3&-1
    \end{bmatrix}
  \end{equation*}
  \begin{equation*}
    E =
    \begin{bmatrix}
      1&-3&5\\
      2&1&-1\\
      1&1&0
    \end{bmatrix}
    \quad
    F =
    \begin{bmatrix}
      1&-1&4\\
      2&-3&6\\
      1&0&1
    \end{bmatrix}.
  \end{equation*}
  Either perform the indicated operation or state the the expression is
  undefined (e.g., because it asks you to add or multiply matrices whose dimenions
  don't allow for multiplication):
  \begin{enumerate}[(a)]
    \item $D-2C$
    \item $A+2E$
    \item $CD$
    \item $DC$
    \item $EF$
    \item $FE$
    \item $AE$
    \item $EA$
    \item $B(C+D)$
    \item $A^{2}$
  \end{enumerate}
\end{problem}

\end{document}