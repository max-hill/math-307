\documentclass[10pt]{article}
% Math Packages
\usepackage{amsmath, mathtools}
\usepackage{amssymb}
\usepackage{amsthm}
\usepackage{amsfonts}
\usepackage{bbm}
\usepackage{breqn}
\usepackage[margin=1in]{geometry}
\usepackage{graphicx}
\usepackage{tikz}
\usetikzlibrary{arrows.meta}
\usetikzlibrary{calc}
\usepackage{forest}
\usepackage{tikz-qtree}
\graphicspath{ {./images/} }
\usepackage{hyperref}
\usepackage[capitalize]{cleveref}
\usepackage[shortlabels]{enumitem}
\usetikzlibrary{arrows,matrix,positioning}
\usepackage{multicol}


% for the pipe symbol
\usepackage[T1]{fontenc}

% Citing theorems by name. (source: https://tex.stackexchange.com/questions/109843/cleveref-and-named-theorems)
\makeatletter
\newcommand{\ncref}[1]{\cref{#1}\mynameref{#1}{\csname r@#1\endcsname}}

\def\mynameref#1#2{%
  \begingroup
  \edef\@mytxt{#2}%
  \edef\@mytst{\expandafter\@thirdoffive\@mytxt}%
  \ifx\@mytst\empty\else
  \space(\nameref{#1})\fi
  \endgroup
}
\makeatother

% Colorful Notes
\usepackage{color} \definecolor{Red}{rgb}{1,0,0} \definecolor{Blue}{rgb}{0,0,1}
\definecolor{Purple}{rgb}{.5,0,.5} \def\red{\color{Red}} \def\blue{\color{Blue}}
\def\gray{\color{gray}} \def\purple{\color{Purple}}
\newcommand{\rnote}[1]{{\red [#1]}} % \rnote{foo} gives '[foo]' in red
\newcommand{\pnote}[1]{{\purple [#1]}} % \pnote{foo} gives '[foo]' in purple
\newcommand{\bnote}[1]{{\blue #1}} % \bnote{foo} gives 'foo' in blue
\newcommand{\gnote}[1]{{\gray #1}} % \gnote{foo} gives 'foo' in gray
\newcommand{\Max}[1]{{\purple [#1]}} % \bnote{foo} then 'foo' is blue


% Claim numbering (the counter restarts after each proof environment)
\newcounter{claimcount}
\setcounter{claimcount}{0}
\newenvironment{claim}{\refstepcounter{claimcount}\par\addvspace{\medskipamount}\noindent\textbf{Claim \arabic{claimcount}:}}{}
\usepackage{etoolbox}
\AtBeginEnvironment{proof}{\setcounter{claimcount}{0}}
\newenvironment{claimproof}{\par\addvspace{\medskipamount}\noindent\textit{Proof of Claim  \arabic{claimcount}.}}{\hfill\ensuremath{\qedsymbol} \tiny{Claim}

  \medskip}
% Add claim support to cleverref
\crefname{claimcount}{Claim}{Claims}


% Math Environments
\newtheorem{theorem}{Theorem}
\newtheorem{assumption}[theorem]{Assumption}
\newtheorem{lemma}[theorem]{Lemma}
\newtheorem{proposition}[theorem]{Proposition}
\newtheorem{corollary}[theorem]{Corollary}
\newtheorem{question}[theorem]{Question}
\theoremstyle{definition}
\newtheorem{definition}[theorem]{Definition}
\newtheorem{remark}[theorem]{Remark}
\newtheorem{example}[theorem]{Example}
\newtheorem{notation}[theorem]{Notation}
\newtheorem{problem}[theorem]{Problem}

% Matrices and Column Vectors. 
\usepackage{stackengine}
\setstackgap{L}{1.0\normalbaselineskip}
\usepackage{tabstackengine}
\setstackEOL{;}% row separator
\setstackTAB{,}% column separator
\setstacktabbedgap{1ex}% inter-column gap
\setstackgap{L}{1.5\normalbaselineskip}% inter-row baselineskip
\let\nmatrix\bracketMatrixstack  %Usage: \nmatrix{1,2,3\4,5,6}
\newcommand\cv[1]{\setstackEOL{,}\bracketMatrixstack{#1}} %usage: \cv{1,2,3}

% Custom Math Coqmmands
\newcommand{\vt}{\vskip 5mm} % vertical space
\newcommand{\fl}{\noindent\textbf} % first line
\newcommand{\Fl}{\vt\noindent\textbf} % first line with space above
\newcommand{\norm}[1]{\left\lVert#1\right\rVert} % norm
\newcommand{\pnorm}[1]{\left\lVert#1\right\rVert_p} % p-norm
\newcommand{\qnorm}[1]{\left\lVert#1\right\rVert_q} % q-norm
\newcommand{\1}[1]{\textbf{1}_{\left[#1\right]}} % indicator function 
\def\limn{\lim_{n\to\infty}} % shortcut for lim as n-> infinity
\def\sumn{\sum_{n=1}^{\infty}} % shortcut for sum from n=1 to infinity
\def\sumkn{\sum_{k=1}^{n}} % shortcut for sum from k=1 to n
\def\sumin{\sum_{i=1}^{n}} % shortcut for sum from i=1 to n
\def\SAs{\sigma\text{-algebras}} % shortcut for $\sigma$-algebras
\def\SA{\sigma\text{-algebra}} % shortcut for $\sigma$-algebra
\def\Ft{\mathcal{F}_t} % time-indexed sigma-algebra (t)
\def\Fs{\mathcal{F}_s} % time-indexed sigma-algebra (s)
\def\F{\mathcal{F}} % sigma-algebra
\def\G{\mathcal{G}} % sigma-algebra
\def\R{\mathbb{R}} % Real numbers
\def\N{\mathbb{N}} % Natural numbers
\def\Z{\mathbb{Z}} % Integers
\def\E{\mathbb{E}} % Expectation
\def\P{\mathbb{P}} % Probability
\def\Q{\mathbb{Q}} % Q probability
\def\dist{\text{dist}} %Text 'dist' for things like 'dist(x,y)'
\newcommand{\indep}{\perp \!\!\! \perp}  %independence symbol
\def\Var{\mathrm{Var}} % Variance
\def\tr{\mathrm{tr}} % trace

% Brackets and Parentheses
\def\[{\left [}
    \def\]{\right ]}
% \def\({\left (}
%   \def\){\right )}



\usepackage{color}
\definecolor{Red}{rgb}{1,0,0}
\definecolor{Blue}{rgb}{0,0,1}
\definecolor{Purple}{rgb}{.5,0,.5}
\def\red{\color{Red}}
\def\blue{\color{Blue}}
\def\gray{\color{gray}}
\def\purple{\color{Purple}}
\definecolor{RoyalBlue}{cmyk}{1, 0.50, 0, 0}
\newcommand{\dempfcolor}[1]{{\color{RoyalBlue}#1}} 
\newcommand{\demph}[1]{\dempfcolor{{\sl #1}}}

% comment exactly one of the following line to show / hide the solutions
% \newcommand{\solution}[1]{{\purple #1}} % uncomment to show the solutions
\newcommand{\solution}[1]{} % uncomment to hide the solutions



\title{Lecture Notes for Math 372: \\Elementary Probability and Statistics}
\date{Last updated: \today}
% \author{mh}

\begin{document}
\begin{center}
  \section*{Math 307: Homework 04}
  \textit{Due Wednesday, October 1 (at the beginning of class)}
\end{center}


% \begin{problem}[The matrix exponential]
%   Recall that for any real number $x$, the \demph{exponential function}
%   $e^{x}$ is defined by the power series
%   \begin{equation}\label{eq:1}
%     e^{x} = \sum_{k=0}^{\infty}\frac{x^{k}}{k!}.
%   \end{equation}
%   \begin{enumerate}[(a)]
%     \item Remarkably, the exponential function is defined even if $x$ is a
%     matrix. In particular, letting $A$ be any square matrix, we can define the
%     \demph{matrix exponential} as
%     \begin{equation*}
%       e^{A} = \sum_{k=0}^{\infty} \frac{A^{k}}{k!}.
%     \end{equation*}
%     Note that $e^{A}$ is a square matrix of the same dimensions as $A$.
%     Compute $e^{A}$ when $A =
%     \begin{bmatrix}
%       2&0\\
%       0&3
%     \end{bmatrix}
%     $. 
%     \item Let $b$ and $t$ be real numbers. By differentiating the series in
%     \cref{eq:1} term-by-term (with $x=bt$), show that
%     \begin{equation*}
%       \frac{d}{dt} \left[ e^{bt}  \right] = b e^{bt}.
%     \end{equation*}
%     \item Let $B$ be any square matrix, and let $t$ be a real number.
%     Following your approach in part (b), show that
%     \begin{equation*}
%       \frac{d}{dt} \left[ e^{Bt} \right]  = B e^{Bt}.
%     \end{equation*}
%   \end{enumerate}
% \end{problem}


\begin{problem}
  Determine which of the following sets of vectors are subspaces of $\R^{2}$.
  \begin{enumerate}[(a)]
    \item All vectors of the form $
    \begin{bmatrix}
      0\\y
    \end{bmatrix}
    $
    \item All vectors of the form $
    \begin{bmatrix}
      x\\3x
    \end{bmatrix}
    $
    \item All vectors of the form
    $
    \begin{bmatrix}
      x\\2-5x
    \end{bmatrix}
    $
    \item All vectors $
    \begin{bmatrix}
      x\\y
    \end{bmatrix}
    $
    where $x+y=0$.
  \end{enumerate}
\end{problem}


\begin{problem}
  A linear system is said to be \demph{homogeneous} if the right hand sides are all zero,
  like this:
  \begin{align*}
    x-y+z&=0\\
    2x+y+2z&=0\\
    3x-5y+3z&=0
  \end{align*}
  Such systems always have at least one solution, namely the ``trivial
  solution'' in which all the variables are zero (i.e., $x=y=z=0$ in this
  case). Using row reduction, find all solutions to the above linear system,
  and include a sketch or plot of the solutions.
\end{problem}
\begin{problem}
  Determine whether the following matrices are invertible
  \begin{enumerate}[(a)]
    \item
    $\begin{bmatrix}
      6&-3\\
      -4&2
    \end{bmatrix}$
    \item
    $\begin{bmatrix}
      5&-1\\
      3&4
    \end{bmatrix}$ 
    \item
    $\begin{bmatrix}
      -1&0&2\\
      1&1&-1\\
      3&0&1
    \end{bmatrix}$ 
  \end{enumerate}
\end{problem}

\begin{problem}
  Let
  \begin{equation*}
    A =
    \begin{bmatrix}
      3&-2\\
      1&4
    \end{bmatrix}
    \quad
    B =
    \begin{bmatrix}
      1&2\\
      -2&3
    \end{bmatrix}.
  \end{equation*}
  \begin{enumerate}[(a)]
    \item Find the determinants of  $A$ and $B$.
    \item Find $\det(AB)$, $\det(A^{-1})$, and $\det(B^{\top}A^{-1})$ without
    finding $AB$, $A^{-1}$, or $B^{\top}A^{-1}$
    \item Show that $\det(A+B)$ is not the same as $\det(A)+\det(B)$.
  \end{enumerate}
\end{problem}

\begin{problem}
  Recall from lecture 10 the vector space
  \begin{equation*}
    C[0,1] = \left\{f:[0,1]\to \R: f\text{ is a continuous function}\right\}.
  \end{equation*}

  Suppose $p\in [0,1]$ is any fixed number in the closed unit interval, and
  define the set $A_{p}$ as
  \begin{equation*}
    A_{p} = \left\{f:[0,1]\to \R: f\text{ is a continuous function with }f(p)=0\right\}.
  \end{equation*}
  Show that $A_{p}$ is a subspace of $C[0,1]$.
\end{problem}

\begin{problem}
  The set of complex numbers $\mathbb{C}$ is defined aso
  $\mathbb{C}=\left\{x+yi: x,y\in \R\right\}$, where $i^{2}=-1$. Does the set of
  complex numbers under addition and scalar multiplciation
  \begin{align*}
    (a+bi)+(c+di)  &= (a+c)+(b+d)i\\
    c(a+bi)&=ca+cbi
  \end{align*}
  (where $a,b,c$ and $d$ are real numbers) form a vector space? If not, why not?
\end{problem}


\begin{problem}
  Determine whether the vectors
  $\begin{bmatrix}
    1\\1\\0
  \end{bmatrix},
  \begin{bmatrix}
    1\\-1\\-1
  \end{bmatrix},
  \begin{bmatrix}
    1\\3\\1
  \end{bmatrix} $
  span $\R^{3}$.
\end{problem}

\begin{problem}
  Determine the conditions on $a,b,c,d\in \R$ such that the following system
  has solutions:
  \begin{align*}
    x+2y-z&=a\\
    x+y-2z&=b\\
    2x+y-3z&=c
  \end{align*}
\end{problem}


\begin{problem}
  Let $V$ be the vector space of real-valued sequences:
  \begin{equation*}
    V = \left\{(a_{1},a_{2},\ldots) : a_{1},a_{2},\ldots\in \R\right\}.
  \end{equation*}
  In this vector space, vector addition and scalar multiplication are defined
  as follows:
  \begin{itemize}
    \item If $a=(a_{1},a_{2},a_{3},\ldots)\in V$ and
    $b=(b_{1},b_{2},b_{3},\ldots)\in V$, define ``vector addition'' as
    \begin{equation*}
      a+b = (a_{1}+b_{1},\ a_{2}+b_{2},\ a_{3}+b_{3},\ \ldots).
    \end{equation*}
    \item  For $a=(a_{1},a_{2},a_{3},\ldots)\in V$ and $c\in \R$, define ``scalar multiplication'' by
    \begin{equation*}
      ca = (ca_{1},\ ca_{2},\ ca_{3},\ \ldots).
    \end{equation*}
  \end{itemize}
  Let $F$ be the following subset of $V$:
  \begin{equation*}
    F = \left\{(a_{1},a_{2},a_{3},\ldots)\in V: a_{n}=a_{n-1}+a_{n-2} \text{ for
        each }n=3,4,5,\ldots\right\}.
  \end{equation*}
  For example, $F$ contains the Fibonacci sequence
  \begin{equation*}
    (a_{1},a_{2},a_{3},a_{4},a_{5},a_{6},a_{7},a_{8},a_{9},a_{10},a_{11},...)= (0, 1, 1, 2, 3, 5, 8, 13, 21, 34, 55,\ldots)
  \end{equation*}
  as well as other ``Fibonacci-like'' sequences.
  
  Verify that $F$ is a linear subspace of $V$. (You may assume that $V$ is a
  vector spaces.)
\end{problem}
\end{document}

% Don't write in starlight, 'Cause the words may come out real