\documentclass[10pt]{article}
% Math Packages
\usepackage{amsmath, mathtools}
\usepackage{amssymb}
\usepackage{amsthm}
\usepackage{amsfonts}
\usepackage{bbm}
\usepackage{breqn}
\usepackage[margin=1in]{geometry}
\usepackage{graphicx}
\usepackage{tikz}
\usetikzlibrary{arrows.meta}
\usetikzlibrary{calc}
\usepackage{forest}
\usepackage{tikz-qtree}
\graphicspath{ {./images/} }
\usepackage{hyperref}
\usepackage[capitalize]{cleveref}
\usepackage[shortlabels]{enumitem}
\usetikzlibrary{arrows,matrix,positioning}
\usepackage{multicol}

% for boxed text
\usepackage[most]{tcolorbox}%

% for the pipe symbol
\usepackage[T1]{fontenc}

% Citing theorems by name. (source: https://tex.stackexchange.com/questions/109843/cleveref-and-named-theorems)
\makeatletter
\newcommand{\ncref}[1]{\cref{#1}\mynameref{#1}{\csname r@#1\endcsname}}

\def\mynameref#1#2{%
  \begingroup
  \edef\@mytxt{#2}%
  \edef\@mytst{\expandafter\@thirdoffive\@mytxt}%
  \ifx\@mytst\empty\else
  \space(\nameref{#1})\fi
  \endgroup
}
\makeatother

% Colorful Notes
\usepackage{color} \definecolor{Red}{rgb}{1,0,0} \definecolor{Blue}{rgb}{0,0,1}
\definecolor{Purple}{rgb}{.5,0,.5} \def\red{\color{Red}} \def\blue{\color{Blue}}
\def\gray{\color{gray}} \def\purple{\color{Purple}}
\newcommand{\rnote}[1]{{\red [#1]}} % \rnote{foo} gives '[foo]' in red
\newcommand{\pnote}[1]{{\purple [#1]}} % \pnote{foo} gives '[foo]' in purple
\newcommand{\bnote}[1]{{\blue #1}} % \bnote{foo} gives 'foo' in blue
\newcommand{\gnote}[1]{{\gray #1}} % \gnote{foo} gives 'foo' in gray
\newcommand{\Max}[1]{{\purple [#1]}} % \bnote{foo} then 'foo' is blue


% Claim numbering (the counter restarts after each proof environment)
\newcounter{claimcount}
\setcounter{claimcount}{0}
\newenvironment{claim}{\refstepcounter{claimcount}\par\addvspace{\medskipamount}\noindent\textbf{Claim \arabic{claimcount}:}}{}
\usepackage{etoolbox}
\AtBeginEnvironment{proof}{\setcounter{claimcount}{0}}
\newenvironment{claimproof}{\par\addvspace{\medskipamount}\noindent\textit{Proof of Claim  \arabic{claimcount}.}}{\hfill\ensuremath{\qedsymbol} \tiny{Claim}

  \medskip}
% Add claim support to cleverref
\crefname{claimcount}{Claim}{Claims}


% Math Environments
\newtheorem{theorem}{Theorem}
\newtheorem{assumption}[theorem]{Assumption}
\newtheorem{lemma}[theorem]{Lemma}
\newtheorem{proposition}[theorem]{Proposition}
\newtheorem{corollary}[theorem]{Corollary}
\newtheorem{question}[theorem]{Question}
\theoremstyle{definition}
\newtheorem{definition}[theorem]{Definition}
\newtheorem{remark}[theorem]{Remark}
\newtheorem{example}[theorem]{Example}
\newtheorem{notation}[theorem]{Notation}
\newtheorem{problem}[theorem]{Problem}

% Matrices and Column Vectors. 
\usepackage{stackengine}
\setstackgap{L}{1.0\normalbaselineskip}
\usepackage{tabstackengine}
\setstackEOL{;}% row separator
\setstackTAB{,}% column separator
\setstacktabbedgap{1ex}% inter-column gap
\setstackgap{L}{1.5\normalbaselineskip}% inter-row baselineskip
\let\nmatrix\bracketMatrixstack  %Usage: \nmatrix{1,2,3\4,5,6}
\newcommand\cv[1]{\setstackEOL{,}\bracketMatrixstack{#1}} %usage: \cv{1,2,3}

% Custom Math Coqmmands
\newcommand{\vt}{\vskip 5mm} % vertical space
\newcommand{\fl}{\noindent\textbf} % first line
\newcommand{\Fl}{\vt\noindent\textbf} % first line with space above
\newcommand{\norm}[1]{\left\lVert#1\right\rVert} % norm
\newcommand{\pnorm}[1]{\left\lVert#1\right\rVert_p} % p-norm
\newcommand{\qnorm}[1]{\left\lVert#1\right\rVert_q} % q-norm
\newcommand{\1}[1]{\textbf{1}_{\left[#1\right]}} % indicator function 
\def\limn{\lim_{n\to\infty}} % shortcut for lim as n-> infinity
\def\sumn{\sum_{n=1}^{\infty}} % shortcut for sum from n=1 to infinity
\def\sumkn{\sum_{k=1}^{n}} % shortcut for sum from k=1 to n
\def\sumin{\sum_{i=1}^{n}} % shortcut for sum from i=1 to n
\def\SAs{\sigma\text{-algebras}} % shortcut for $\sigma$-algebras
\def\SA{\sigma\text{-algebra}} % shortcut for $\sigma$-algebra
\def\Ft{\mathcal{F}_t} % time-indexed sigma-algebra (t)
\def\Fs{\mathcal{F}_s} % time-indexed sigma-algebra (s)
\def\F{\mathcal{F}} % sigma-algebra
\def\G{\mathcal{G}} % sigma-algebra
\def\R{\mathbb{R}} % Real numbers
\def\N{\mathbb{N}} % Natural numbers
\def\Z{\mathbb{Z}} % Integers
\def\E{\mathbb{E}} % Expectation
\def\P{\mathbb{P}} % Probability
\def\Q{\mathbb{Q}} % Q probability
\def\dist{\text{dist}} %Text 'dist' for things like 'dist(x,y)'
\newcommand{\indep}{\perp \!\!\! \perp}  %independence symbol
\def\Var{\mathrm{Var}} % Variance
\def\tr{\mathrm{tr}} % trace

% Brackets and Parentheses
\def\[{\left [}
    \def\]{\right ]}
% \def\({\left (}
%   \def\){\right )}



\usepackage{color}
\definecolor{Red}{rgb}{1,0,0}
\definecolor{Blue}{rgb}{0,0,1}
\definecolor{Purple}{rgb}{.5,0,.5}
\def\red{\color{Red}}
\def\blue{\color{Blue}}
\def\gray{\color{gray}}
\def\purple{\color{Purple}}
\definecolor{RoyalBlue}{cmyk}{1, 0.50, 0, 0}
\newcommand{\dempfcolor}[1]{{\color{RoyalBlue}#1}} 
\newcommand{\demph}[1]{\dempfcolor{{\sl #1}}}

% comment exactly one of the following line to show / hide the solutions
% \newcommand{\solution}[1]{{\purple #1}} % uncomment to show the solutions
\newcommand{\solution}[1]{} % uncomment to hide the solutions



\title{Lecture Notes for Math 372: \\Elementary Probability and Statistics}
\date{Last updated: \today}
% \author{mh}

\begin{document}
\begin{center}
  \section*{Math 307: Homework 09}
  \textit{Due Wednesday, November 12. Material up to and including 5.4.}
\end{center}



\begin{problem}
  Let $A=
  \begin{bmatrix}
    1&-1\\
    2&4
  \end{bmatrix}$.
  The \demph{trace} of $A$ is the sum of its main diagonal entries. Find the
  eigenvalues of $A$, verify that trace equals the sum of the eigenvalues, and
  that the determinant equals their product. 
\end{problem}

\begin{problem}
  Find the eigenvectors and eigenvalues of
  \begin{equation*}
    A =
    \begin{bmatrix}
      3&4&2\\
      0&1&2\\
      0&0&0
    \end{bmatrix}
    \quad \text{and} \quad
    B =
    \begin{bmatrix}
      0&0&2\\
      0&2&0\\
      2&0&0
    \end{bmatrix}
  \end{equation*}
  Check that $\lambda_{1}+\lambda_{2}+\lambda_{3}$ equals the trace and that
  $\lambda_{1}\lambda_{2}\lambda_{3}$ equals the determinant. (Here
  $\lambda_{1},\lambda_{2},\lambda_{3}$ are the eigenvalues of the matrix.)
\end{problem}


% \begin{problem}
%   Every permutation matrix leaves $x=(1,1,\ldots,1)^{\top}$ unchanged. Then
%   $\lambda=1$. Find two more eigenvalues for these permutation matrices:
%   \begin{equation*}
%     P =
%     \begin{bmatrix}
%       0&1&0\\
%       0&0&1\\
%       1&0&0
%     \end{bmatrix}
%     \quad \text{and} \quad
%     Q =
%     \begin{bmatrix}
%       0&0&1\\
%       0&1&0\\
%       1&0&0
%     \end{bmatrix}
%   \end{equation*}
% \end{problem}

\begin{problem}
  Find the rank and nullspace of
  \begin{equation*}
    A=
    \begin{bmatrix}
      0&0&1\\
      0&0&1\\
      1&1&1
    \end{bmatrix}
    \quad \text{and} \quad
    B =
    \begin{bmatrix}
      0&0&1&2\\
      0&0&1&2\\
      1&1&1&0
    \end{bmatrix}
  \end{equation*}
\end{problem}

\begin{problem}
  Suppose all vectors $x$ in the unit square $0 \leq x_{1}\leq 1$, $0 \leq
  x_{2}\leq 1$ are transformed to $Ax$. (Here $A$ is a $2\times 2$ matrix).
  \begin{enumerate}[(a)]
    \item What is the shape of the transformed region (all $Ax$)?
    \item For which matrices $A$ is that region a square?
    \item For which $A$ is it a line?
    \item For which $A$ is the new area still $1$?
  \end{enumerate}
\end{problem}


\begin{problem}
  Given an eigenvalue $\lambda$ of matrix $A$, the \demph{eigenspace} of
  $\lambda$ is the subspace $NS(\lambda I-A)$. For the following matrices,
  find the eigenvalues, and find a basis for each eigenspace.
  \begin{enumerate}[(a)]
    \item $
    \begin{bmatrix}
      3&0\\
      8&-1
    \end{bmatrix}
    $
    \item $
    \begin{bmatrix}
      4&-4\\1&0
    \end{bmatrix}
    $
    \item $
    \begin{bmatrix}
      4&0&1\\-2&1&0\\-2&0&1
    \end{bmatrix}
    $
  \end{enumerate}
\end{problem}


\begin{problem}
  Let $A,B,C$ be square matrices. Prove the following statements:
  \begin{enumerate}[(a)]
    \item If $B$ is similar to $A$, then $A$ is similar to $B$. (``Similarity
    is reflexive'')
    \item If $A$ is similar to $B$, and $B$ is similar to $C$, then $A$ is
    similar to $C$. (``Similarity is transitive'')
    \item If $A,B$ are similar, then $\det(A)=\det(B)$.
    \item If $A$ is invertible, then $B$ is invertible and $B^{-1}$ is
    similar to $A^{-1}$.
  \end{enumerate}
\end{problem}

\newpage
\begin{problem}
  \textit{(Hint for this problem: refer to the examples in section 5.1 of the textbook.)} Let
  $T:\R^{3}\to P_{2}$ be a linear transformation satisfying
  \begin{equation*}
    T
    \begin{bmatrix}
      1\\1\\0
    \end{bmatrix}
    = x^{2}+x,\quad
    T
    \begin{bmatrix}
      1\\-1\\1
    \end{bmatrix}
    = x^{2}-x+1, \quad
    T
    \begin{bmatrix}
      0\\1\\1
    \end{bmatrix}
    =x+1.
  \end{equation*}
  \begin{enumerate}[(a)]
    \item Find $T
    \begin{bmatrix}
      1\\0\\0
    \end{bmatrix}.
    $
    \item Find $T
    \begin{bmatrix}
      a\\b\\c
    \end{bmatrix}.
    $   
  \end{enumerate}
\end{problem}

\begin{problem}
  \begin{enumerate}[(a)]
    \item []
    \item Construct a matrix whose nullspace contains the vector $x=
    \begin{bmatrix}
      1\\1\\2
    \end{bmatrix}
    $
    \item Construct a matrix whose column space is spanned by $
    \begin{bmatrix}
      1\\1\\2
    \end{bmatrix}
    $ and whose row space is spanned by $
    \begin{bmatrix}
      1&5
    \end{bmatrix}
    $.
    \item How can you construct a matrix that transforms the coordinate vectors
    $e_{1},e_{2},e_{3}$ into three given vectors $v_{1},v_{2},v_{3}$. When
    will that matrix be invertible?
  \end{enumerate}
\end{problem}

\begin{problem}
  Let $T:\R^{3}\to \R^{3}$ be given by
  \begin{equation*}
    T
    \begin{bmatrix}
      x_{1}\\x_{2}\\x_{3}
    \end{bmatrix}
    =
    \begin{bmatrix}
      17x_{1}-8x_{2}-12x_{3}\\
      16x_{1}-7x_{2}-12x_{3}\\
      16x_{1}-8x_{2}-11x_{3}
    \end{bmatrix}.
  \end{equation*} 
  \begin{enumerate}[(a)]
    \item Find $[T]_{\alpha}^{\alpha}$, where $\alpha$ is the standard basis
    for $\R^{3}$.
    \item Let $\beta=\left\{\begin{bmatrix}
        1\\1\\1
      \end{bmatrix},
      \begin{bmatrix}
        1\\2\\0
      \end{bmatrix},
      \begin{bmatrix}
        1\\-1\\2
      \end{bmatrix}\right\}.$
    Find the change of basis matrix from $\alpha$ to $\beta$.
    \item Find the change of basis matrix from $\beta$ to $\alpha$. 
    \item Find $[T]_{\beta}^{\beta}$.
    \item Find $[v]_{\beta}$ for $v=
    \begin{bmatrix}
      2\\-1\\4
    \end{bmatrix}.$ 
    \item Find $[T(v)]_{\beta}$.
    \item Use the result of part (f) to find $T(v)$.
  \end{enumerate}
\end{problem}

\begin{problem}
  A linear transformation from $V$ to $W$ has an \textit{inverse} from $W$ to
  $V$ when the range of is all of $W$ and the kernel contains only $v=0$. Why
  are these transformations not invertible?
  \begin{enumerate}[(a)]
    \item $T(v_{1},v_{2})=(v_{2},v_{2})$, \quad $W=\R^{2}$
    \item $T(v_{1},v_{2})=(v_{1},v_{2},v_{1}+v_{2})$, \quad $W=\R^{2}$
    \item $T(v_{1},v_{2})=v_{1}$, \quad $W=\R^{1}$
  \end{enumerate}
\end{problem}

\begin{problem}
  Suppose $T:V \to W$ is a linear transformation.
  \begin{enumerate}[(a)]
    \item Show that if $v\in V$ and $u\in \ker(T)$, then $T(u+v)=T(v)$.
    \item Show that if $u,v\in V$ such that $T(u)=T(v)$, then $u-v$ is in
    $\ker(T)$.
    \item Use part (b) to deduce that if $\ker(T)=\left\{0\right\}$, then $T$
    is injective. [Note: a function $f$ is \demph{injective} if
    $x_{1}\neq x_{2}$ implies $f(x_{1})\neq f(x_{2})$].
  \end{enumerate}
\end{problem}

% \begin{problem}
%   Show that the set of linear transformations from a vector space $V$ to a
%   vector space $W$ is a vector space.
% \end{problem}



\begin{problem}[An application of eigenvalues]
  If $f:\R^{2}\to \R$ is a twice-differentiable function of two variables,
  then the \demph{Hessian matrix} of $f$ is the matrix
  \begin{equation*}
    H(x,y) =
    \begin{bmatrix}
      f_{xx}(x,y)&f_{xy}(x,y)\\
      f_{yx}(x,y)&f_{yy}(x,y)
    \end{bmatrix}.
  \end{equation*}
  
  In this problem, we'll use the following theorem:
{\footnotesize
  \begin{tcolorbox}[colframe=black, colback=gray!5, arc=2mm, boxrule=0.8pt,
    breakable, title=The Second Derivative Test,   coltitle=black,colbacktitle=black!10]
    Let $f:\R^{2} \to \R$ be a twice-differentiable function of two variables,
    and let $(a,b)$ be a critical point of $f$. Let $\lambda_{1}$ and $\lambda_{2}$
    be the eigenvalues of $H(a,b)$. Then:
    \begin{itemize}
      \item If $\lambda_{1},\lambda_{2}$ are both positive, then $f$ has a minimum at
      $(x,y)=(a,b) $.
      \item If $\lambda_{1},\lambda_{2}$ are both negative, then $f$ has a minimum at
      $(x,y)=(a,b) $.
      \item If one of $\lambda_{1},\lambda_{2}$ is negative and the other
      positive, then  $f$ has a saddle point at $(x,y)=(a,b)$.
      \item If one or more eigenvalue is zero, then the test is
      inconclusive.
    \end{itemize}
  \end{tcolorbox}}
  For parts (a) and (b), find the critical points of $f(x,y)$ and classify them
  using the second derivative test stated here.
  \begin{enumerate}[(a)]
    \item Let $f(x,y) = x^{2}+4xy+y^{2}$. 
    \item Let $f(x,y) = x^{3}-3x+y^{2}-4y$
  \end{enumerate}
\end{problem}

\begin{problem}
  For the following matrices, find the eigenvalues, and find a basis for each
  eigenspace.
  \begin{equation*}
    A=\begin{bmatrix}
      1&1&0\\0&1&-1\\0&0&-2
    \end{bmatrix}
    \quad \text{and} \quad 
    B=\begin{bmatrix}
      4&0&0\\1&4&0\\0&1&4
    \end{bmatrix}
    % $
    % \begin{bmatrix}
    %   5&6&2\\0&-1&-8\\1&0&-2
    % \end{bmatrix}
    % $
  \end{equation*}
\end{problem}

\begin{problem}
  Find a basis for the following subspace of $\R^{4}$:
  \begin{enumerate}[(a)]
    \item The vectors for which $x_{1} = 2x_{4}$.
    \item The vectors for which $x_{1}+x_{2}+x_{3}=0$ and $x_{3}+x_{4}=0$.
    \item The subspace spanned by $
    \begin{bmatrix}
      1\\1\\1\\1
    \end{bmatrix},
    \begin{bmatrix}
      1\\2\\3\\4
    \end{bmatrix},
    \begin{bmatrix}
      2\\3\\4\\5
    \end{bmatrix}
    $
  \end{enumerate}
\end{problem}


\begin{problem}
  Suppose a linear transformation $T$ transforms $(1,1)$ to $(2,2)$ and
  $(2,0)$ to $(0,0)$. Find $T(v)$ when
  \begin{enumerate}[(a)]
    \item $v=(2,2)$
    \item $v=(3,1)$
    \item $v=(-1,1)$
    \item $v=(a,b)$
  \end{enumerate}
\end{problem}


\end{document}

% Don't write in starlight, 'Cause the words may come out real