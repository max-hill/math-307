\documentclass[10pt]{article}
\usepackage[margin=1in]{geometry}
\usepackage{graphicx}
\graphicspath{{./images/}}

\usepackage{amsmath, amssymb, amsthm, amsfonts, mathtools, bbm, breqn}
\usepackage{enumitem}       % for shortlabels
\usepackage{multicol}       % multiple columns
\usepackage{abraces}        % asymmetric braces
\usepackage{skull}
\usepackage{tikz}
\usetikzlibrary{arrows.meta, arrows, calc, matrix, positioning}

\usepackage{hyperref}
\usepackage[capitalize]{cleveref}


% Citing theorems by name. (source: https://tex.stackexchange.com/questions/109843/cleveref-and-named-theorems)
\makeatletter
\newcommand{\ncref}[1]{\cref{#1}\mynameref{#1}{\csname r@#1\endcsname}}

\def\mynameref#1#2{%
  \begingroup
  \edef\@mytxt{#2}%
  \edef\@mytst{\expandafter\@thirdoffive\@mytxt}%
  \ifx\@mytst\empty\else
  \space(\nameref{#1})\fi
  \endgroup
}
\makeatother

% for the pipe symbol
\usepackage[T1]{fontenc}

\usepackage{xcolor} % Enables a broader range of colors

% Define custom colors
\definecolor{RoyalBlue}{cmyk}{1, 0.50, 0, 0}

% Note commands
\definecolor{Red}{rgb}{1,0,0}
\definecolor{Blue}{rgb}{0,0,1}
\definecolor{Purple}{rgb}{.75,0,.25}
\newcommand{\rnote}[1]{\textcolor{Red}{[#1]}}       % Red note
\newcommand{\pnote}[1]{\textcolor{Purple}{[#1]}}    % Purple note
\newcommand{\bnote}[1]{\textcolor{Blue}{#1}}        % Blue text
\newcommand{\Max}[1]{\pnote{#1}}                    % Alias for purple note

% Emphasized text
\newcommand{\demph}[1]{\textcolor{RoyalBlue}{\slshape #1}} % Slanted RoyalBlue text


% Claim numbering (the counter restarts after each proof environment)
\newcounter{claimcount}
\setcounter{claimcount}{0}
\newenvironment{claim}{\refstepcounter{claimcount}\par\addvspace{\medskipamount}\noindent\textbf{Claim \arabic{claimcount}:}}{}
\usepackage{etoolbox}
\AtBeginEnvironment{proof}{\setcounter{claimcount}{0}}
\newenvironment{claimproof}{\par\addvspace{\medskipamount}\noindent\textit{Proof of Claim  \arabic{claimcount}.}}{\hfill\ensuremath{\qedsymbol} \tiny{Claim}

  \medskip}
% Add claim support to cleverref
\crefname{claimcount}{Claim}{Claims}


% Math Environments
\newtheorem{theorem}{Theorem}
\newtheorem{assumption}[theorem]{Assumption}
\newtheorem{lemma}[theorem]{Lemma}
\newtheorem{proposition}[theorem]{Proposition}
\newtheorem{corollary}[theorem]{Corollary}
\newtheorem{question}[theorem]{Question}
\theoremstyle{definition}
\newtheorem{definition}[theorem]{Definition}
\newtheorem{remark}[theorem]{Remark}
\newtheorem{example}[theorem]{Example}
\newtheorem{notation}[theorem]{Notation}
\newtheorem{problem}[theorem]{Problem}

% Redefine the Example environment to include "End of example [number]"
\makeatletter
\let\oldexample\example
\renewenvironment{example}
{\begin{oldexample}}
  {\par\smallskip\hfill   End of Example~\theexample. $\square$    \par\end{oldexample}}
\makeatother

% Custom Math Commands
\newcommand{\vt}{\vspace{5mm}}                       % Vertical space
\newcommand{\fl}[1]{\noindent\textbf{#1}}            % Bold first line
\newcommand{\Fl}[1]{\vspace{5mm}\noindent\textbf{#1}}% Bold first line with space above
\newcommand{\norm}[1]{\left\lVert #1 \right\rVert}   % Norm

% Common math symbols
\newcommand{\R}{\mathbb{R}}           % Real numbers
\newcommand{\N}{\mathbb{N}}           % Natural numbers
\newcommand{\C}{\mathbb{C}}           % Complex numbers
\newcommand{\Z}{\mathbb{Z}}           % Integers
\newcommand{\Q}{\mathbb{Q}}           % Rational numbers
\newcommand{\E}{\mathbb{E}}           % Expectation
\renewcommand{\P}{\mathbb{P}}         % Probability (renamed to avoid \P clash)
\newcommand{\indep}{\perp\!\!\!\perp} % Independence symbol

% Operators
\newcommand{\Var}{\mathrm{Var}}   % Variance
\newcommand{\tr}{\mathrm{tr}}     % Trace
\newcommand{\dist}{\mathrm{dist}} % Distance


\usepackage{biblatex}
\addbibresource{refs.bib}

\title{Lecture Notes for Math 307:\\Linear Algebra and Differential Equations}
\author{Instructor: Max Hill (Fall 2025)}
\date{Last updated: \today}
\begin{document}

\maketitle
\tableofcontents

\section*{About this document}

These lecture notes were prepared by Max Hill for a 16-week linear algebra course  (MATH
307) at University of Hawaii at Manoa in Fall 2025.

The textbook used is \textit{Linear Algebra and Differential Equations} (2002)
by G.~Peterson S.~Sochacki, in which we cover primarily Chapters 1,2,5, and 6


\newpage
\setcounter{section}{-1}
\section{Tentative Course Outline}

\begin{itemize}
  \item \textbf{Weeks 1-3: Matrices and determinants.} \textit{(Systems of linear
  equations, matrices, matrix operations, inverse matrices, special matrices
  and their properties, and determinants.)}
  \begin{itemize} 
    \item Section 1.1: Systems of Linear Equations
    \item Section 1.2: Matrices and Matrix Operations
    \item Section 1.3: Inverses of Matrices
    \item Section 1.4: Special Matrices and Additional Properties of Matrices
    \item Section 1.5: Determinants
    \item Section 1.6: Further Properties of Determinants
    \item Section 1.7: Proofs of Theorems on Determinants
  \end{itemize}

  \item \textbf{Weeks 4-6: Vector spaces.} \textit{(Vector spaces, subspaces, spanning
  sets, linear independence, bases, dimension, null space, row and column
  spaces, Wronskian.)}
  \begin{itemize}
    \item Section 2.1: Vector Spaces
    \item Section 2.2: Subspaces and Spanning Sets
    \item Section 2.3: Linear Independence and Bases
    \item Section 2.4: Dimension; Nullspace, Rowspace, and Column Space
    \item Section 2.5: Wronskians
  \end{itemize}
  \item \textbf{Weeks 7-11: Linear transformations, spectral theory.} \textit{(Linear
  transformation, eigenvalues and eigenvectors, algebra of linear
  transformations, matrices for linear transformations, eigenvalues and
  eigenvectors, similar matrices, diagonalization, Jordan normal form.)}
  \begin{itemize}
    \item Section 5.1: Linear Transformations
    \item Section 5.2: The Algebra of Linear Transformations
    \item Section 5.3: Matrices for Linear Transformations
    \item Section 5.4: Eigenvalues and Eigenvectors of Matrices
    \item Section 5.5: Similar Matrices, Diagonalization, and Jordan Canonical Form
    \item Section 5.6: Eigenvectors and Eigenvalues of Linear Transformations
  \end{itemize}
  \item \textbf{Midterm Exam}
  \item \textbf{Weeks 12-14: Systems of differential equations.} \textit{(Theory of
  systems of linear differential equations, homogeneous systems with constant
  coefficients, the diagonalizable case, nondiagonalizable case,
  nonhomogeneous linear systems, applications to $2\times 2$ and $3\times 3$
  systems of nonlinear differential equations.)}
  \begin{itemize}
    \item Section 6.1: The THeory of Systems of Linear Differential Equations
    \item Section 6.2: Homogenous Systems with Constant Coefficients: The
    Diagonalizable Case
    \item Section 6.3: Homogenous Systems with Constant Coefficients: The
    Nondiagonalizable Case
    \item Section 6.4: Nonhomogeneous Linear Systems
    \item Section 6.6: Applications Involving Systems of Linear Differential Equations
    \item Section 6.7: $2\times 2$ Systems of Nonlinear Differential Equations
  \end{itemize}
  \item \textbf{Weeks 14-16: Other stuff if time allows.} \textit{(Converting differential
  equations to first order systems (section 6.5), linearization of $2 \times 2$ nonlinear
  systems (???), stability and instability (section 6.7), predator-prey
  equations (section 6.7.1).)}
  \item \textbf{Final Exam}
\end{itemize}

\newpage
\section{2025-08-25 | Week 01 | Lecture 01}
\textit{This lecture is based on section 1.1 in the textbook.}


\printbibliography
\end{document}